
% Default to the notebook output style

    


% Inherit from the specified cell style.




    
\documentclass[11pt]{article}

    
    
    \usepackage[T1]{fontenc}
    % Nicer default font (+ math font) than Computer Modern for most use cases
    \usepackage{mathpazo}

    % Basic figure setup, for now with no caption control since it's done
    % automatically by Pandoc (which extracts ![](path) syntax from Markdown).
    \usepackage{graphicx}
    % We will generate all images so they have a width \maxwidth. This means
    % that they will get their normal width if they fit onto the page, but
    % are scaled down if they would overflow the margins.
    \makeatletter
    \def\maxwidth{\ifdim\Gin@nat@width>\linewidth\linewidth
    \else\Gin@nat@width\fi}
    \makeatother
    \let\Oldincludegraphics\includegraphics
    % Set max figure width to be 80% of text width, for now hardcoded.
    \renewcommand{\includegraphics}[1]{\Oldincludegraphics[width=.8\maxwidth]{#1}}
    % Ensure that by default, figures have no caption (until we provide a
    % proper Figure object with a Caption API and a way to capture that
    % in the conversion process - todo).
    \usepackage{caption}
    \DeclareCaptionLabelFormat{nolabel}{}
    \captionsetup{labelformat=nolabel}

    \usepackage{adjustbox} % Used to constrain images to a maximum size 
    \usepackage{xcolor} % Allow colors to be defined
    \usepackage{enumerate} % Needed for markdown enumerations to work
    \usepackage{geometry} % Used to adjust the document margins
    \usepackage{amsmath} % Equations
    \usepackage{amssymb} % Equations
    \usepackage{textcomp} % defines textquotesingle
    % Hack from http://tex.stackexchange.com/a/47451/13684:
    \AtBeginDocument{%
        \def\PYZsq{\textquotesingle}% Upright quotes in Pygmentized code
    }
    \usepackage{upquote} % Upright quotes for verbatim code
    \usepackage{eurosym} % defines \euro
    \usepackage[mathletters]{ucs} % Extended unicode (utf-8) support
    \usepackage[utf8x]{inputenc} % Allow utf-8 characters in the tex document
    \usepackage{fancyvrb} % verbatim replacement that allows latex
    \usepackage{grffile} % extends the file name processing of package graphics 
                         % to support a larger range 
    % The hyperref package gives us a pdf with properly built
    % internal navigation ('pdf bookmarks' for the table of contents,
    % internal cross-reference links, web links for URLs, etc.)
    \usepackage{hyperref}
    \usepackage{longtable} % longtable support required by pandoc >1.10
    \usepackage{booktabs}  % table support for pandoc > 1.12.2
    \usepackage[inline]{enumitem} % IRkernel/repr support (it uses the enumerate* environment)
    \usepackage[normalem]{ulem} % ulem is needed to support strikethroughs (\sout)
                                % normalem makes italics be italics, not underlines
    

    
    
    % Colors for the hyperref package
    \definecolor{urlcolor}{rgb}{0,.145,.698}
    \definecolor{linkcolor}{rgb}{.71,0.21,0.01}
    \definecolor{citecolor}{rgb}{.12,.54,.11}

    % ANSI colors
    \definecolor{ansi-black}{HTML}{3E424D}
    \definecolor{ansi-black-intense}{HTML}{282C36}
    \definecolor{ansi-red}{HTML}{E75C58}
    \definecolor{ansi-red-intense}{HTML}{B22B31}
    \definecolor{ansi-green}{HTML}{00A250}
    \definecolor{ansi-green-intense}{HTML}{007427}
    \definecolor{ansi-yellow}{HTML}{DDB62B}
    \definecolor{ansi-yellow-intense}{HTML}{B27D12}
    \definecolor{ansi-blue}{HTML}{208FFB}
    \definecolor{ansi-blue-intense}{HTML}{0065CA}
    \definecolor{ansi-magenta}{HTML}{D160C4}
    \definecolor{ansi-magenta-intense}{HTML}{A03196}
    \definecolor{ansi-cyan}{HTML}{60C6C8}
    \definecolor{ansi-cyan-intense}{HTML}{258F8F}
    \definecolor{ansi-white}{HTML}{C5C1B4}
    \definecolor{ansi-white-intense}{HTML}{A1A6B2}

    % commands and environments needed by pandoc snippets
    % extracted from the output of `pandoc -s`
    \providecommand{\tightlist}{%
      \setlength{\itemsep}{0pt}\setlength{\parskip}{0pt}}
    \DefineVerbatimEnvironment{Highlighting}{Verbatim}{commandchars=\\\{\}}
    % Add ',fontsize=\small' for more characters per line
    \newenvironment{Shaded}{}{}
    \newcommand{\KeywordTok}[1]{\textcolor[rgb]{0.00,0.44,0.13}{\textbf{{#1}}}}
    \newcommand{\DataTypeTok}[1]{\textcolor[rgb]{0.56,0.13,0.00}{{#1}}}
    \newcommand{\DecValTok}[1]{\textcolor[rgb]{0.25,0.63,0.44}{{#1}}}
    \newcommand{\BaseNTok}[1]{\textcolor[rgb]{0.25,0.63,0.44}{{#1}}}
    \newcommand{\FloatTok}[1]{\textcolor[rgb]{0.25,0.63,0.44}{{#1}}}
    \newcommand{\CharTok}[1]{\textcolor[rgb]{0.25,0.44,0.63}{{#1}}}
    \newcommand{\StringTok}[1]{\textcolor[rgb]{0.25,0.44,0.63}{{#1}}}
    \newcommand{\CommentTok}[1]{\textcolor[rgb]{0.38,0.63,0.69}{\textit{{#1}}}}
    \newcommand{\OtherTok}[1]{\textcolor[rgb]{0.00,0.44,0.13}{{#1}}}
    \newcommand{\AlertTok}[1]{\textcolor[rgb]{1.00,0.00,0.00}{\textbf{{#1}}}}
    \newcommand{\FunctionTok}[1]{\textcolor[rgb]{0.02,0.16,0.49}{{#1}}}
    \newcommand{\RegionMarkerTok}[1]{{#1}}
    \newcommand{\ErrorTok}[1]{\textcolor[rgb]{1.00,0.00,0.00}{\textbf{{#1}}}}
    \newcommand{\NormalTok}[1]{{#1}}
    
    % Additional commands for more recent versions of Pandoc
    \newcommand{\ConstantTok}[1]{\textcolor[rgb]{0.53,0.00,0.00}{{#1}}}
    \newcommand{\SpecialCharTok}[1]{\textcolor[rgb]{0.25,0.44,0.63}{{#1}}}
    \newcommand{\VerbatimStringTok}[1]{\textcolor[rgb]{0.25,0.44,0.63}{{#1}}}
    \newcommand{\SpecialStringTok}[1]{\textcolor[rgb]{0.73,0.40,0.53}{{#1}}}
    \newcommand{\ImportTok}[1]{{#1}}
    \newcommand{\DocumentationTok}[1]{\textcolor[rgb]{0.73,0.13,0.13}{\textit{{#1}}}}
    \newcommand{\AnnotationTok}[1]{\textcolor[rgb]{0.38,0.63,0.69}{\textbf{\textit{{#1}}}}}
    \newcommand{\CommentVarTok}[1]{\textcolor[rgb]{0.38,0.63,0.69}{\textbf{\textit{{#1}}}}}
    \newcommand{\VariableTok}[1]{\textcolor[rgb]{0.10,0.09,0.49}{{#1}}}
    \newcommand{\ControlFlowTok}[1]{\textcolor[rgb]{0.00,0.44,0.13}{\textbf{{#1}}}}
    \newcommand{\OperatorTok}[1]{\textcolor[rgb]{0.40,0.40,0.40}{{#1}}}
    \newcommand{\BuiltInTok}[1]{{#1}}
    \newcommand{\ExtensionTok}[1]{{#1}}
    \newcommand{\PreprocessorTok}[1]{\textcolor[rgb]{0.74,0.48,0.00}{{#1}}}
    \newcommand{\AttributeTok}[1]{\textcolor[rgb]{0.49,0.56,0.16}{{#1}}}
    \newcommand{\InformationTok}[1]{\textcolor[rgb]{0.38,0.63,0.69}{\textbf{\textit{{#1}}}}}
    \newcommand{\WarningTok}[1]{\textcolor[rgb]{0.38,0.63,0.69}{\textbf{\textit{{#1}}}}}
    
    
    % Define a nice break command that doesn't care if a line doesn't already
    % exist.
    \def\br{\hspace*{\fill} \\* }
    % Math Jax compatability definitions
    \def\gt{>}
    \def\lt{<}
    % Document parameters
    \title{Assignment}
    
    
    

    % Pygments definitions
    
\makeatletter
\def\PY@reset{\let\PY@it=\relax \let\PY@bf=\relax%
    \let\PY@ul=\relax \let\PY@tc=\relax%
    \let\PY@bc=\relax \let\PY@ff=\relax}
\def\PY@tok#1{\csname PY@tok@#1\endcsname}
\def\PY@toks#1+{\ifx\relax#1\empty\else%
    \PY@tok{#1}\expandafter\PY@toks\fi}
\def\PY@do#1{\PY@bc{\PY@tc{\PY@ul{%
    \PY@it{\PY@bf{\PY@ff{#1}}}}}}}
\def\PY#1#2{\PY@reset\PY@toks#1+\relax+\PY@do{#2}}

\expandafter\def\csname PY@tok@m\endcsname{\def\PY@tc##1{\textcolor[rgb]{0.40,0.40,0.40}{##1}}}
\expandafter\def\csname PY@tok@sx\endcsname{\def\PY@tc##1{\textcolor[rgb]{0.00,0.50,0.00}{##1}}}
\expandafter\def\csname PY@tok@s1\endcsname{\def\PY@tc##1{\textcolor[rgb]{0.73,0.13,0.13}{##1}}}
\expandafter\def\csname PY@tok@nf\endcsname{\def\PY@tc##1{\textcolor[rgb]{0.00,0.00,1.00}{##1}}}
\expandafter\def\csname PY@tok@k\endcsname{\let\PY@bf=\textbf\def\PY@tc##1{\textcolor[rgb]{0.00,0.50,0.00}{##1}}}
\expandafter\def\csname PY@tok@fm\endcsname{\def\PY@tc##1{\textcolor[rgb]{0.00,0.00,1.00}{##1}}}
\expandafter\def\csname PY@tok@kr\endcsname{\let\PY@bf=\textbf\def\PY@tc##1{\textcolor[rgb]{0.00,0.50,0.00}{##1}}}
\expandafter\def\csname PY@tok@mf\endcsname{\def\PY@tc##1{\textcolor[rgb]{0.40,0.40,0.40}{##1}}}
\expandafter\def\csname PY@tok@sb\endcsname{\def\PY@tc##1{\textcolor[rgb]{0.73,0.13,0.13}{##1}}}
\expandafter\def\csname PY@tok@gt\endcsname{\def\PY@tc##1{\textcolor[rgb]{0.00,0.27,0.87}{##1}}}
\expandafter\def\csname PY@tok@kc\endcsname{\let\PY@bf=\textbf\def\PY@tc##1{\textcolor[rgb]{0.00,0.50,0.00}{##1}}}
\expandafter\def\csname PY@tok@vm\endcsname{\def\PY@tc##1{\textcolor[rgb]{0.10,0.09,0.49}{##1}}}
\expandafter\def\csname PY@tok@kp\endcsname{\def\PY@tc##1{\textcolor[rgb]{0.00,0.50,0.00}{##1}}}
\expandafter\def\csname PY@tok@mi\endcsname{\def\PY@tc##1{\textcolor[rgb]{0.40,0.40,0.40}{##1}}}
\expandafter\def\csname PY@tok@vc\endcsname{\def\PY@tc##1{\textcolor[rgb]{0.10,0.09,0.49}{##1}}}
\expandafter\def\csname PY@tok@mh\endcsname{\def\PY@tc##1{\textcolor[rgb]{0.40,0.40,0.40}{##1}}}
\expandafter\def\csname PY@tok@o\endcsname{\def\PY@tc##1{\textcolor[rgb]{0.40,0.40,0.40}{##1}}}
\expandafter\def\csname PY@tok@cm\endcsname{\let\PY@it=\textit\def\PY@tc##1{\textcolor[rgb]{0.25,0.50,0.50}{##1}}}
\expandafter\def\csname PY@tok@vi\endcsname{\def\PY@tc##1{\textcolor[rgb]{0.10,0.09,0.49}{##1}}}
\expandafter\def\csname PY@tok@ne\endcsname{\let\PY@bf=\textbf\def\PY@tc##1{\textcolor[rgb]{0.82,0.25,0.23}{##1}}}
\expandafter\def\csname PY@tok@ch\endcsname{\let\PY@it=\textit\def\PY@tc##1{\textcolor[rgb]{0.25,0.50,0.50}{##1}}}
\expandafter\def\csname PY@tok@nn\endcsname{\let\PY@bf=\textbf\def\PY@tc##1{\textcolor[rgb]{0.00,0.00,1.00}{##1}}}
\expandafter\def\csname PY@tok@sh\endcsname{\def\PY@tc##1{\textcolor[rgb]{0.73,0.13,0.13}{##1}}}
\expandafter\def\csname PY@tok@kn\endcsname{\let\PY@bf=\textbf\def\PY@tc##1{\textcolor[rgb]{0.00,0.50,0.00}{##1}}}
\expandafter\def\csname PY@tok@ni\endcsname{\let\PY@bf=\textbf\def\PY@tc##1{\textcolor[rgb]{0.60,0.60,0.60}{##1}}}
\expandafter\def\csname PY@tok@cp\endcsname{\def\PY@tc##1{\textcolor[rgb]{0.74,0.48,0.00}{##1}}}
\expandafter\def\csname PY@tok@nl\endcsname{\def\PY@tc##1{\textcolor[rgb]{0.63,0.63,0.00}{##1}}}
\expandafter\def\csname PY@tok@sc\endcsname{\def\PY@tc##1{\textcolor[rgb]{0.73,0.13,0.13}{##1}}}
\expandafter\def\csname PY@tok@err\endcsname{\def\PY@bc##1{\setlength{\fboxsep}{0pt}\fcolorbox[rgb]{1.00,0.00,0.00}{1,1,1}{\strut ##1}}}
\expandafter\def\csname PY@tok@mb\endcsname{\def\PY@tc##1{\textcolor[rgb]{0.40,0.40,0.40}{##1}}}
\expandafter\def\csname PY@tok@il\endcsname{\def\PY@tc##1{\textcolor[rgb]{0.40,0.40,0.40}{##1}}}
\expandafter\def\csname PY@tok@go\endcsname{\def\PY@tc##1{\textcolor[rgb]{0.53,0.53,0.53}{##1}}}
\expandafter\def\csname PY@tok@gu\endcsname{\let\PY@bf=\textbf\def\PY@tc##1{\textcolor[rgb]{0.50,0.00,0.50}{##1}}}
\expandafter\def\csname PY@tok@kd\endcsname{\let\PY@bf=\textbf\def\PY@tc##1{\textcolor[rgb]{0.00,0.50,0.00}{##1}}}
\expandafter\def\csname PY@tok@w\endcsname{\def\PY@tc##1{\textcolor[rgb]{0.73,0.73,0.73}{##1}}}
\expandafter\def\csname PY@tok@cs\endcsname{\let\PY@it=\textit\def\PY@tc##1{\textcolor[rgb]{0.25,0.50,0.50}{##1}}}
\expandafter\def\csname PY@tok@c\endcsname{\let\PY@it=\textit\def\PY@tc##1{\textcolor[rgb]{0.25,0.50,0.50}{##1}}}
\expandafter\def\csname PY@tok@nt\endcsname{\let\PY@bf=\textbf\def\PY@tc##1{\textcolor[rgb]{0.00,0.50,0.00}{##1}}}
\expandafter\def\csname PY@tok@bp\endcsname{\def\PY@tc##1{\textcolor[rgb]{0.00,0.50,0.00}{##1}}}
\expandafter\def\csname PY@tok@gp\endcsname{\let\PY@bf=\textbf\def\PY@tc##1{\textcolor[rgb]{0.00,0.00,0.50}{##1}}}
\expandafter\def\csname PY@tok@vg\endcsname{\def\PY@tc##1{\textcolor[rgb]{0.10,0.09,0.49}{##1}}}
\expandafter\def\csname PY@tok@gs\endcsname{\let\PY@bf=\textbf}
\expandafter\def\csname PY@tok@na\endcsname{\def\PY@tc##1{\textcolor[rgb]{0.49,0.56,0.16}{##1}}}
\expandafter\def\csname PY@tok@sr\endcsname{\def\PY@tc##1{\textcolor[rgb]{0.73,0.40,0.53}{##1}}}
\expandafter\def\csname PY@tok@se\endcsname{\let\PY@bf=\textbf\def\PY@tc##1{\textcolor[rgb]{0.73,0.40,0.13}{##1}}}
\expandafter\def\csname PY@tok@dl\endcsname{\def\PY@tc##1{\textcolor[rgb]{0.73,0.13,0.13}{##1}}}
\expandafter\def\csname PY@tok@kt\endcsname{\def\PY@tc##1{\textcolor[rgb]{0.69,0.00,0.25}{##1}}}
\expandafter\def\csname PY@tok@ge\endcsname{\let\PY@it=\textit}
\expandafter\def\csname PY@tok@s2\endcsname{\def\PY@tc##1{\textcolor[rgb]{0.73,0.13,0.13}{##1}}}
\expandafter\def\csname PY@tok@ss\endcsname{\def\PY@tc##1{\textcolor[rgb]{0.10,0.09,0.49}{##1}}}
\expandafter\def\csname PY@tok@gh\endcsname{\let\PY@bf=\textbf\def\PY@tc##1{\textcolor[rgb]{0.00,0.00,0.50}{##1}}}
\expandafter\def\csname PY@tok@s\endcsname{\def\PY@tc##1{\textcolor[rgb]{0.73,0.13,0.13}{##1}}}
\expandafter\def\csname PY@tok@gd\endcsname{\def\PY@tc##1{\textcolor[rgb]{0.63,0.00,0.00}{##1}}}
\expandafter\def\csname PY@tok@gr\endcsname{\def\PY@tc##1{\textcolor[rgb]{1.00,0.00,0.00}{##1}}}
\expandafter\def\csname PY@tok@nc\endcsname{\let\PY@bf=\textbf\def\PY@tc##1{\textcolor[rgb]{0.00,0.00,1.00}{##1}}}
\expandafter\def\csname PY@tok@cpf\endcsname{\let\PY@it=\textit\def\PY@tc##1{\textcolor[rgb]{0.25,0.50,0.50}{##1}}}
\expandafter\def\csname PY@tok@ow\endcsname{\let\PY@bf=\textbf\def\PY@tc##1{\textcolor[rgb]{0.67,0.13,1.00}{##1}}}
\expandafter\def\csname PY@tok@sd\endcsname{\let\PY@it=\textit\def\PY@tc##1{\textcolor[rgb]{0.73,0.13,0.13}{##1}}}
\expandafter\def\csname PY@tok@nv\endcsname{\def\PY@tc##1{\textcolor[rgb]{0.10,0.09,0.49}{##1}}}
\expandafter\def\csname PY@tok@gi\endcsname{\def\PY@tc##1{\textcolor[rgb]{0.00,0.63,0.00}{##1}}}
\expandafter\def\csname PY@tok@nd\endcsname{\def\PY@tc##1{\textcolor[rgb]{0.67,0.13,1.00}{##1}}}
\expandafter\def\csname PY@tok@no\endcsname{\def\PY@tc##1{\textcolor[rgb]{0.53,0.00,0.00}{##1}}}
\expandafter\def\csname PY@tok@si\endcsname{\let\PY@bf=\textbf\def\PY@tc##1{\textcolor[rgb]{0.73,0.40,0.53}{##1}}}
\expandafter\def\csname PY@tok@nb\endcsname{\def\PY@tc##1{\textcolor[rgb]{0.00,0.50,0.00}{##1}}}
\expandafter\def\csname PY@tok@sa\endcsname{\def\PY@tc##1{\textcolor[rgb]{0.73,0.13,0.13}{##1}}}
\expandafter\def\csname PY@tok@c1\endcsname{\let\PY@it=\textit\def\PY@tc##1{\textcolor[rgb]{0.25,0.50,0.50}{##1}}}
\expandafter\def\csname PY@tok@mo\endcsname{\def\PY@tc##1{\textcolor[rgb]{0.40,0.40,0.40}{##1}}}

\def\PYZbs{\char`\\}
\def\PYZus{\char`\_}
\def\PYZob{\char`\{}
\def\PYZcb{\char`\}}
\def\PYZca{\char`\^}
\def\PYZam{\char`\&}
\def\PYZlt{\char`\<}
\def\PYZgt{\char`\>}
\def\PYZsh{\char`\#}
\def\PYZpc{\char`\%}
\def\PYZdl{\char`\$}
\def\PYZhy{\char`\-}
\def\PYZsq{\char`\'}
\def\PYZdq{\char`\"}
\def\PYZti{\char`\~}
% for compatibility with earlier versions
\def\PYZat{@}
\def\PYZlb{[}
\def\PYZrb{]}
\makeatother


    % Exact colors from NB
    \definecolor{incolor}{rgb}{0.0, 0.0, 0.5}
    \definecolor{outcolor}{rgb}{0.545, 0.0, 0.0}



    
    % Prevent overflowing lines due to hard-to-break entities
    \sloppy 
    % Setup hyperref package
    \hypersetup{
      breaklinks=true,  % so long urls are correctly broken across lines
      colorlinks=true,
      urlcolor=urlcolor,
      linkcolor=linkcolor,
      citecolor=citecolor,
      }
    % Slightly bigger margins than the latex defaults
    
    \geometry{verbose,tmargin=1in,bmargin=1in,lmargin=1in,rmargin=1in}
    
    

    \begin{document}
    
    
    \maketitle
    
    

    
    \section{Assignment}\label{assignment}

    \subsection{Question 1:}\label{question-1}

Write a program that calculates and prints the value according to the
given formula:

Q = Square root of {[}(2 * C * D)/H{]}

Following are the fixed values of C and H:

C is 50. H is 30.

D is the variable whose values should be input to your program in a
comma-separated sequence.

\paragraph{Example}\label{example}

Let us assume the following comma separated input sequence is given to
the program:

100,150,180

The output of the program should be:

18,22,24

\paragraph{Hints:}\label{hints}

If the output received is in decimal form, it should be rounded off to
its nearest value (for example, if the output received is 26.0, it
should be printed as 26).

    \subsubsection{Version 1}\label{version-1}

    \begin{Verbatim}[commandchars=\\\{\}]
{\color{incolor}In [{\color{incolor}1}]:} \PY{k+kn}{import} \PY{n+nn}{math}                                              \PY{c+c1}{\PYZsh{}import necessary libraries }
\end{Verbatim}


    \begin{Verbatim}[commandchars=\\\{\}]
{\color{incolor}In [{\color{incolor} }]:} \PY{c+c1}{\PYZsh{}define function to implement formula}
        \PY{k}{def} \PY{n+nf}{squareRoot}\PY{p}{(}\PY{n+nb}{int}\PY{p}{)}\PY{p}{:}
            \PY{k}{return} \PY{p}{(}\PY{n}{math}\PY{o}{.}\PY{n}{floor}\PY{p}{(}\PY{n}{math}\PY{o}{.}\PY{n}{sqrt}\PY{p}{(}\PY{p}{(}\PY{l+m+mi}{2} \PY{o}{*} \PY{n}{C} \PY{o}{*} \PY{n}{D}\PY{p}{)} \PY{o}{/} \PY{n}{H}\PY{p}{)}\PY{p}{)}\PY{p}{)}
\end{Verbatim}


    \paragraph{\texorpdfstring{Formula :
\[ Q = {\sqrt {2 * C * D\over H}} \]}{Formula :  Q = \{\textbackslash{}sqrt \{2 * C * D\textbackslash{}over H\}\} }}\label{formula-q-sqrt-2-c-dover-h}

    \begin{Verbatim}[commandchars=\\\{\}]
{\color{incolor}In [{\color{incolor}19}]:} \PY{c+c1}{\PYZsh{}initialize pre defined values}
         \PY{n}{C} \PY{o}{=} \PY{l+m+mi}{50}
         \PY{n}{H} \PY{o}{=} \PY{l+m+mi}{30}
         
         \PY{n}{D} \PY{o}{=} \PY{n+nb}{int}\PY{p}{(}\PY{n+nb}{input}\PY{p}{(}\PY{l+s+s2}{\PYZdq{}}\PY{l+s+s2}{enter number : }\PY{l+s+s2}{\PYZdq{}}\PY{p}{)}\PY{p}{)}                       \PY{c+c1}{\PYZsh{} input the variable D}
         \PY{n+nb}{print}\PY{p}{(}\PY{l+s+s2}{\PYZdq{}}\PY{l+s+s2}{The answer is }\PY{l+s+si}{\PYZob{}\PYZcb{}}\PY{l+s+s2}{ }\PY{l+s+s2}{\PYZdq{}}\PY{o}{.}\PY{n}{format}\PY{p}{(}\PY{n}{squareRoot}\PY{p}{(}\PY{n}{D}\PY{p}{)}\PY{p}{)}\PY{p}{)}        \PY{c+c1}{\PYZsh{} print answer by calling the defined function}
\end{Verbatim}


    \begin{Verbatim}[commandchars=\\\{\}]
enter number : 100
The answer is 18 

    \end{Verbatim}

    \subsubsection{Version 2}\label{version-2}

    \begin{Verbatim}[commandchars=\\\{\}]
{\color{incolor}In [{\color{incolor}18}]:} \PY{k+kn}{import} \PY{n+nn}{math}                                              \PY{c+c1}{\PYZsh{}import necessary libraries }
\end{Verbatim}


    \paragraph{\texorpdfstring{Formula :
\[ Q = {\sqrt {2 * C * D\over H}} \]}{Formula :  Q = \{\textbackslash{}sqrt \{2 * C * D\textbackslash{}over H\}\} }}\label{formula-q-sqrt-2-c-dover-h}

    \begin{Verbatim}[commandchars=\\\{\}]
{\color{incolor}In [{\color{incolor}21}]:} \PY{c+c1}{\PYZsh{}\PYZsh{}initialize pre\PYZhy{}defined values}
         \PY{n}{C} \PY{o}{=} \PY{l+m+mi}{50}
         \PY{n}{H} \PY{o}{=} \PY{l+m+mi}{30}
         
         \PY{n}{D} \PY{o}{=} \PY{n+nb}{int}\PY{p}{(}\PY{n+nb}{input}\PY{p}{(}\PY{l+s+s2}{\PYZdq{}}\PY{l+s+s2}{enter number : }\PY{l+s+s2}{\PYZdq{}}\PY{p}{)}\PY{p}{)}                        \PY{c+c1}{\PYZsh{} input the variable D}
         \PY{n}{answer} \PY{o}{=} \PY{n}{math}\PY{o}{.}\PY{n}{floor}\PY{p}{(}\PY{n}{math}\PY{o}{.}\PY{n}{sqrt}\PY{p}{(}\PY{p}{(}\PY{l+m+mi}{2} \PY{o}{*} \PY{n}{C} \PY{o}{*} \PY{n}{D}\PY{p}{)} \PY{o}{/} \PY{n}{H}\PY{p}{)}\PY{p}{)}          \PY{c+c1}{\PYZsh{} calculate answer by passing through provided formula}
         \PY{n+nb}{print}\PY{p}{(}\PY{l+s+s2}{\PYZdq{}}\PY{l+s+s2}{the answer is }\PY{l+s+si}{\PYZob{}\PYZcb{}}\PY{l+s+s2}{ }\PY{l+s+s2}{\PYZdq{}}\PY{o}{.}\PY{n}{format}\PY{p}{(}\PY{n}{answer}\PY{p}{)}\PY{p}{)}                \PY{c+c1}{\PYZsh{} print answer }
\end{Verbatim}


    \begin{Verbatim}[commandchars=\\\{\}]
enter number : 150
the answer is 22 

    \end{Verbatim}

    \subsection{Question 2}\label{question-2}

    Write a program which takes 2 digits, X,Y as input and generates a
2-dimensional array. The element value in the i-th row and j-th column
of the array should be i*j.

Note: i=0,1.., X-1; j=0,1,¡ Y-1.

\paragraph{Example}\label{example}

Suppose the following inputs are given to the program:

3,5

Then, the output of the program should be:

{[}{[}0, 0, 0, 0, 0{]}, {[}0, 1, 2, 3, 4{]}, {[}0, 2, 4, 6, 8{]}{]}

    \subsubsection{Version 1}\label{version-1}

    \begin{Verbatim}[commandchars=\\\{\}]
{\color{incolor}In [{\color{incolor}4}]:} \PY{c+c1}{\PYZsh{}define function to display matrix by accepting variables}
        \PY{k}{def} \PY{n+nf}{displayMatrix}\PY{p}{(}\PY{n}{x}\PY{p}{,}\PY{n}{y}\PY{p}{)}\PY{p}{:}
            \PY{n}{dim\PYZus{}array} \PY{o}{=} \PY{p}{[}\PY{p}{[}\PY{l+m+mi}{0} \PY{k}{for} \PY{n}{j} \PY{o+ow}{in} \PY{n+nb}{range}\PY{p}{(}\PY{n}{y}\PY{p}{)}\PY{p}{]} \PY{k}{for} \PY{n}{i} \PY{o+ow}{in} \PY{n+nb}{range}\PY{p}{(}\PY{n}{x}\PY{p}{)}\PY{p}{]}        \PY{c+c1}{\PYZsh{} define array}
            \PY{k}{for} \PY{n}{i} \PY{o+ow}{in} \PY{n+nb}{range}\PY{p}{(}\PY{n}{x}\PY{p}{)}\PY{p}{:}                                           \PY{c+c1}{\PYZsh{} use recurvise loop to reach entered numbers  }
                \PY{k}{for} \PY{n}{j} \PY{o+ow}{in} \PY{n+nb}{range}\PY{p}{(}\PY{n}{y}\PY{p}{)}\PY{p}{:}
                    \PY{n}{dim\PYZus{}array}\PY{p}{[}\PY{n}{i}\PY{p}{]}\PY{p}{[}\PY{n}{j}\PY{p}{]} \PY{o}{=} \PY{n}{i} \PY{o}{*} \PY{n}{j}                              \PY{c+c1}{\PYZsh{} allocate array by multiplying the row and columns}
            \PY{k}{return}\PY{p}{(}\PY{n}{dim\PYZus{}array}\PY{p}{)}                                            \PY{c+c1}{\PYZsh{} return array}
\end{Verbatim}


    \begin{Verbatim}[commandchars=\\\{\}]
{\color{incolor}In [{\color{incolor}5}]:} \PY{n}{x} \PY{o}{=} \PY{n+nb}{int}\PY{p}{(}\PY{n+nb}{input}\PY{p}{(}\PY{l+s+s2}{\PYZdq{}}\PY{l+s+s2}{Input the number of rows : }\PY{l+s+s2}{\PYZdq{}}\PY{p}{)}\PY{p}{)}                    \PY{c+c1}{\PYZsh{} input the number of rows}
        \PY{n}{y} \PY{o}{=} \PY{n+nb}{int}\PY{p}{(}\PY{n+nb}{input}\PY{p}{(}\PY{l+s+s2}{\PYZdq{}}\PY{l+s+s2}{Input the number of columns : }\PY{l+s+s2}{\PYZdq{}}\PY{p}{)}\PY{p}{)}                 \PY{c+c1}{\PYZsh{} input number of columns}
        
        \PY{n+nb}{print}\PY{p}{(}\PY{l+s+s2}{\PYZdq{}}\PY{l+s+s2}{the 2\PYZhy{}Dimensional array is : }\PY{l+s+si}{\PYZob{}\PYZcb{}}\PY{l+s+s2}{\PYZdq{}}\PY{o}{.} \PY{n+nb}{format}\PY{p}{(}\PY{n}{displayMatrix}\PY{p}{(}\PY{n}{x}\PY{p}{,}\PY{n}{y}\PY{p}{)}\PY{p}{)}\PY{p}{)}    \PY{c+c1}{\PYZsh{} display the array}
\end{Verbatim}


    \begin{Verbatim}[commandchars=\\\{\}]
Input the number of rows : 3
Input the number of columns : 5
the 2-Dimensional array is : [[0, 0, 0, 0, 0], [0, 1, 2, 3, 4], [0, 2, 4, 6, 8]]

    \end{Verbatim}

    \subsubsection{Verision 2}\label{verision-2}

    \begin{Verbatim}[commandchars=\\\{\}]
{\color{incolor}In [{\color{incolor}3}]:} \PY{n}{row\PYZus{}num} \PY{o}{=} \PY{n+nb}{int}\PY{p}{(}\PY{n+nb}{input}\PY{p}{(}\PY{l+s+s2}{\PYZdq{}}\PY{l+s+s2}{Input number of rows: }\PY{l+s+s2}{\PYZdq{}}\PY{p}{)}\PY{p}{)}                           \PY{c+c1}{\PYZsh{} input rows}
        \PY{n}{col\PYZus{}num} \PY{o}{=} \PY{n+nb}{int}\PY{p}{(}\PY{n+nb}{input}\PY{p}{(}\PY{l+s+s2}{\PYZdq{}}\PY{l+s+s2}{Input number of columns: }\PY{l+s+s2}{\PYZdq{}}\PY{p}{)}\PY{p}{)}                        \PY{c+c1}{\PYZsh{} input columns}
        \PY{n}{multi\PYZus{}list} \PY{o}{=} \PY{p}{[}\PY{p}{[}\PY{l+m+mi}{0} \PY{k}{for} \PY{n}{col} \PY{o+ow}{in} \PY{n+nb}{range}\PY{p}{(}\PY{n}{col\PYZus{}num}\PY{p}{)}\PY{p}{]} \PY{k}{for} \PY{n}{row} \PY{o+ow}{in} \PY{n+nb}{range}\PY{p}{(}\PY{n}{row\PYZus{}num}\PY{p}{)}\PY{p}{]}   \PY{c+c1}{\PYZsh{} allocate array}
        
        \PY{k}{for} \PY{n}{row} \PY{o+ow}{in} \PY{n+nb}{range}\PY{p}{(}\PY{n}{row\PYZus{}num}\PY{p}{)}\PY{p}{:}                                               \PY{c+c1}{\PYZsh{} recursive loop to reach entered numbers  }
            \PY{k}{for} \PY{n}{col} \PY{o+ow}{in} \PY{n+nb}{range}\PY{p}{(}\PY{n}{col\PYZus{}num}\PY{p}{)}\PY{p}{:}                                     
                \PY{n}{multi\PYZus{}list}\PY{p}{[}\PY{n}{row}\PY{p}{]}\PY{p}{[}\PY{n}{col}\PY{p}{]}\PY{o}{=} \PY{n}{row}\PY{o}{*}\PY{n}{col}                                    \PY{c+c1}{\PYZsh{} allocate array by multiplying the row and columns}
        
        \PY{n+nb}{print}\PY{p}{(}\PY{l+s+s2}{\PYZdq{}}\PY{l+s+s2}{the 2\PYZhy{}Dimensional array is : }\PY{l+s+si}{\PYZob{}\PYZcb{}}\PY{l+s+s2}{\PYZdq{}}\PY{o}{.}\PY{n}{format}\PY{p}{(}\PY{n}{multi\PYZus{}list}\PY{p}{)}\PY{p}{)}              \PY{c+c1}{\PYZsh{} display the array }
\end{Verbatim}


    \begin{Verbatim}[commandchars=\\\{\}]
Input number of rows: 3
Input number of columns: 5
the 2-Dimensional array is : [[0, 0, 0, 0, 0], [0, 1, 2, 3, 4], [0, 2, 4, 6, 8]]

    \end{Verbatim}

    \subsection{Question 3}\label{question-3}

    \subsubsection{Note: problem similar to Question
2}\label{note-problem-similar-to-question-2}

    Write a program which takes 2 digits, X,Y as input and generates a
2-dimensional array. The element value in the i-th row and j-th column
of the array should be i*j.

Note: i=0,1.., X-1; j=0,1,¡ Y-1.

\paragraph{Example}\label{example}

Suppose the following inputs are given to the program:

3,5

Then, the output of the program should be:

{[}{[}0, 0, 0, 0, 0{]}, {[}0, 1, 2, 3, 4{]}, {[}0, 2, 4, 6, 8{]}{]}

    \subsubsection{Version 1}\label{version-1}

    \begin{Verbatim}[commandchars=\\\{\}]
{\color{incolor}In [{\color{incolor}4}]:} \PY{c+c1}{\PYZsh{}define function to display matrix by accepting variables}
        \PY{k}{def} \PY{n+nf}{displayMatrix}\PY{p}{(}\PY{n}{x}\PY{p}{,}\PY{n}{y}\PY{p}{)}\PY{p}{:}
            \PY{n}{dim\PYZus{}array} \PY{o}{=} \PY{p}{[}\PY{p}{[}\PY{l+m+mi}{0} \PY{k}{for} \PY{n}{j} \PY{o+ow}{in} \PY{n+nb}{range}\PY{p}{(}\PY{n}{y}\PY{p}{)}\PY{p}{]} \PY{k}{for} \PY{n}{i} \PY{o+ow}{in} \PY{n+nb}{range}\PY{p}{(}\PY{n}{x}\PY{p}{)}\PY{p}{]}        \PY{c+c1}{\PYZsh{} define array}
            \PY{k}{for} \PY{n}{i} \PY{o+ow}{in} \PY{n+nb}{range}\PY{p}{(}\PY{n}{x}\PY{p}{)}\PY{p}{:}                                           \PY{c+c1}{\PYZsh{} use recurvise loop to reach entered numbers  }
                \PY{k}{for} \PY{n}{j} \PY{o+ow}{in} \PY{n+nb}{range}\PY{p}{(}\PY{n}{y}\PY{p}{)}\PY{p}{:}
                    \PY{n}{dim\PYZus{}array}\PY{p}{[}\PY{n}{i}\PY{p}{]}\PY{p}{[}\PY{n}{j}\PY{p}{]} \PY{o}{=} \PY{n}{i} \PY{o}{*} \PY{n}{j}                              \PY{c+c1}{\PYZsh{} allocate array by multiplying the row and columns}
            \PY{k}{return}\PY{p}{(}\PY{n}{dim\PYZus{}array}\PY{p}{)}                                            \PY{c+c1}{\PYZsh{} return array}
\end{Verbatim}


    \begin{Verbatim}[commandchars=\\\{\}]
{\color{incolor}In [{\color{incolor}5}]:} \PY{n}{x} \PY{o}{=} \PY{n+nb}{int}\PY{p}{(}\PY{n+nb}{input}\PY{p}{(}\PY{l+s+s2}{\PYZdq{}}\PY{l+s+s2}{Input the number of rows : }\PY{l+s+s2}{\PYZdq{}}\PY{p}{)}\PY{p}{)}                    \PY{c+c1}{\PYZsh{} input the number of rows}
        \PY{n}{y} \PY{o}{=} \PY{n+nb}{int}\PY{p}{(}\PY{n+nb}{input}\PY{p}{(}\PY{l+s+s2}{\PYZdq{}}\PY{l+s+s2}{Input the number of columns : }\PY{l+s+s2}{\PYZdq{}}\PY{p}{)}\PY{p}{)}                 \PY{c+c1}{\PYZsh{} input number of columns}
        
        \PY{n+nb}{print}\PY{p}{(}\PY{l+s+s2}{\PYZdq{}}\PY{l+s+s2}{the 2\PYZhy{}Dimensional array is : }\PY{l+s+si}{\PYZob{}\PYZcb{}}\PY{l+s+s2}{\PYZdq{}}\PY{o}{.} \PY{n+nb}{format}\PY{p}{(}\PY{n}{displayMatrix}\PY{p}{(}\PY{n}{x}\PY{p}{,}\PY{n}{y}\PY{p}{)}\PY{p}{)}\PY{p}{)}    \PY{c+c1}{\PYZsh{} display the array}
\end{Verbatim}


    \begin{Verbatim}[commandchars=\\\{\}]
Input the number of rows : 3
Input the number of columns : 5
the 2-Dimensional array is : [[0, 0, 0, 0, 0], [0, 1, 2, 3, 4], [0, 2, 4, 6, 8]]

    \end{Verbatim}

    \subsubsection{Verision 2}\label{verision-2}

    \begin{Verbatim}[commandchars=\\\{\}]
{\color{incolor}In [{\color{incolor}3}]:} \PY{n}{row\PYZus{}num} \PY{o}{=} \PY{n+nb}{int}\PY{p}{(}\PY{n+nb}{input}\PY{p}{(}\PY{l+s+s2}{\PYZdq{}}\PY{l+s+s2}{Input number of rows: }\PY{l+s+s2}{\PYZdq{}}\PY{p}{)}\PY{p}{)}                           \PY{c+c1}{\PYZsh{} input rows}
        \PY{n}{col\PYZus{}num} \PY{o}{=} \PY{n+nb}{int}\PY{p}{(}\PY{n+nb}{input}\PY{p}{(}\PY{l+s+s2}{\PYZdq{}}\PY{l+s+s2}{Input number of columns: }\PY{l+s+s2}{\PYZdq{}}\PY{p}{)}\PY{p}{)}                        \PY{c+c1}{\PYZsh{} input columns}
        \PY{n}{multi\PYZus{}list} \PY{o}{=} \PY{p}{[}\PY{p}{[}\PY{l+m+mi}{0} \PY{k}{for} \PY{n}{col} \PY{o+ow}{in} \PY{n+nb}{range}\PY{p}{(}\PY{n}{col\PYZus{}num}\PY{p}{)}\PY{p}{]} \PY{k}{for} \PY{n}{row} \PY{o+ow}{in} \PY{n+nb}{range}\PY{p}{(}\PY{n}{row\PYZus{}num}\PY{p}{)}\PY{p}{]}   \PY{c+c1}{\PYZsh{} allocate array}
        
        \PY{k}{for} \PY{n}{row} \PY{o+ow}{in} \PY{n+nb}{range}\PY{p}{(}\PY{n}{row\PYZus{}num}\PY{p}{)}\PY{p}{:}                                               \PY{c+c1}{\PYZsh{} recursive loop to reach entered numbers  }
            \PY{k}{for} \PY{n}{col} \PY{o+ow}{in} \PY{n+nb}{range}\PY{p}{(}\PY{n}{col\PYZus{}num}\PY{p}{)}\PY{p}{:}                                     
                \PY{n}{multi\PYZus{}list}\PY{p}{[}\PY{n}{row}\PY{p}{]}\PY{p}{[}\PY{n}{col}\PY{p}{]}\PY{o}{=} \PY{n}{row}\PY{o}{*}\PY{n}{col}                                    \PY{c+c1}{\PYZsh{} allocate array by multiplying the row and columns}
        
        \PY{n+nb}{print}\PY{p}{(}\PY{l+s+s2}{\PYZdq{}}\PY{l+s+s2}{the 2\PYZhy{}Dimensional array is : }\PY{l+s+si}{\PYZob{}\PYZcb{}}\PY{l+s+s2}{\PYZdq{}}\PY{o}{.}\PY{n}{format}\PY{p}{(}\PY{n}{multi\PYZus{}list}\PY{p}{)}\PY{p}{)}              \PY{c+c1}{\PYZsh{} display the array }
\end{Verbatim}


    \begin{Verbatim}[commandchars=\\\{\}]
Input number of rows: 3
Input number of columns: 5
the 2-Dimensional array is : [[0, 0, 0, 0, 0], [0, 1, 2, 3, 4], [0, 2, 4, 6, 8]]

    \end{Verbatim}

    \subsection{Question 4}\label{question-4}

    Write a program that accepts a sequence of whitespace separated words as
input and prints the words after removing all duplicate words and
sorting them alphanumerically.

Suppose the following input is supplied to the program:

hello world and practice makes perfect and hello world again

Then, the output should be:

again and hello makes perfect practice world

\paragraph{Hints:}\label{hints}

We use set container to remove duplicated data automatically and then
use sorted() to sort the data.

    \subsubsection{Version 1}\label{version-1}

    \begin{Verbatim}[commandchars=\\\{\}]
{\color{incolor}In [{\color{incolor}19}]:} \PY{n}{sequencial\PYZus{}sentence} \PY{o}{=} \PY{n+nb}{input}\PY{p}{(}\PY{l+s+s2}{\PYZdq{}}\PY{l+s+s2}{Enter a sentence seperated by whitespaces : }\PY{l+s+s2}{\PYZdq{}}\PY{p}{)}                 \PY{c+c1}{\PYZsh{} input sequence of words }
         \PY{n}{sequencial\PYZus{}sentence\PYZus{}spilt} \PY{o}{=} \PY{n}{sequencial\PYZus{}sentence}\PY{o}{.}\PY{n}{split}\PY{p}{(}\PY{l+s+s1}{\PYZsq{}}\PY{l+s+s1}{ }\PY{l+s+s1}{\PYZsq{}}\PY{p}{)}                                  \PY{c+c1}{\PYZsh{} split the words based on whitespaces }
         \PY{n}{seq\PYZus{}set} \PY{o}{=} \PY{n+nb}{set}\PY{p}{(}\PY{n}{sequencial\PYZus{}sentence\PYZus{}spilt}\PY{p}{)}                                                    \PY{c+c1}{\PYZsh{} arrange them alphanumerically}
         \PY{n+nb}{print}\PY{p}{(}\PY{l+s+s1}{\PYZsq{}}\PY{l+s+se}{\PYZbs{}n}\PY{l+s+s1}{The output after removing duplicate words and sorting them alphanumerically is : }\PY{l+s+se}{\PYZbs{}n}\PY{l+s+s1}{\PYZsq{}}\PY{p}{,}\PY{l+s+s1}{\PYZsq{}}\PY{l+s+s1}{ }\PY{l+s+s1}{\PYZsq{}}\PY{o}{.}\PY{n}{join}\PY{p}{(}\PY{n+nb}{sorted}\PY{p}{(}\PY{n}{seq\PYZus{}set}\PY{p}{)}\PY{p}{)}\PY{p}{)}    \PY{c+c1}{\PYZsh{} display the arranged sentence}
\end{Verbatim}


    \begin{Verbatim}[commandchars=\\\{\}]
Enter a sentence seperated by whitespaces : hello world and practice makes perfect and hello world again

The output after removing duplicate words and sorting them alphanumerically is : 
 again and hello makes perfect practice world

    \end{Verbatim}

    \subsubsection{Version 2}\label{version-2}

    \begin{Verbatim}[commandchars=\\\{\}]
{\color{incolor}In [{\color{incolor}23}]:} \PY{n}{seq} \PY{o}{=} \PY{n+nb}{input}\PY{p}{(}\PY{l+s+s2}{\PYZdq{}}\PY{l+s+s2}{Enter a sentence seperated by whitespaces : }\PY{l+s+s2}{\PYZdq{}}\PY{p}{)}    \PY{c+c1}{\PYZsh{} input sequence of words           }
         \PY{n}{words} \PY{o}{=} \PY{p}{[}\PY{n}{word} \PY{k}{for} \PY{n}{word} \PY{o+ow}{in} \PY{n}{seq}\PY{o}{.}\PY{n}{split}\PY{p}{(}\PY{l+s+s2}{\PYZdq{}}\PY{l+s+s2}{ }\PY{l+s+s2}{\PYZdq{}}\PY{p}{)}\PY{p}{]}                      \PY{c+c1}{\PYZsh{} split the words based on whitespaces }
         \PY{n+nb}{print}\PY{p}{(}\PY{l+s+s1}{\PYZsq{}}\PY{l+s+se}{\PYZbs{}n}\PY{l+s+s1}{The output after removing duplicate words and sorting them alphanumerically is : }\PY{l+s+se}{\PYZbs{}n}\PY{l+s+s1}{\PYZsq{}}\PY{p}{,}\PY{l+s+s2}{\PYZdq{}}\PY{l+s+s2}{ }\PY{l+s+s2}{\PYZdq{}}\PY{o}{.}\PY{n}{join}\PY{p}{(}\PY{n+nb}{sorted}\PY{p}{(}\PY{n+nb}{list}\PY{p}{(}\PY{n+nb}{set}\PY{p}{(}\PY{n}{words}\PY{p}{)}\PY{p}{)}\PY{p}{)}\PY{p}{)}\PY{p}{)}     \PY{c+c1}{\PYZsh{} arrange them alphanumerically and display the output }
\end{Verbatim}


    \begin{Verbatim}[commandchars=\\\{\}]
Enter a sentence seperated by whitespaces : hello world and practice makes perfect and hello world again

The output after removing duplicate words and sorting them alphanumerically is : 
 again and hello makes perfect practice world

    \end{Verbatim}

    \subsection{Question 5}\label{question-5}

    Write a program which accepts a sequence of comma separated 4 digit
binary numbers as its input and then check whether they are divisible by
5 or not. The numbers that are divisible by 5 are to be printed in a
comma separated sequence.

\paragraph{Example:}\label{example}

0100,0011,1010,1001

Then the output should be:

1010

    \subsubsection{Version 1}\label{version-1}

    \begin{Verbatim}[commandchars=\\\{\}]
{\color{incolor}In [{\color{incolor}4}]:} \PY{n}{bi\PYZus{}num} \PY{o}{=} \PY{p}{[}\PY{n}{i} \PY{k}{for} \PY{n}{i} \PY{o+ow}{in} \PY{n+nb}{input}\PY{p}{(}\PY{p}{)}\PY{o}{.}\PY{n}{split}\PY{p}{(}\PY{l+s+s1}{\PYZsq{}}\PY{l+s+s1}{,}\PY{l+s+s1}{\PYZsq{}}\PY{p}{)}\PY{p}{]}                       \PY{c+c1}{\PYZsh{} input sequence and split at \PYZsq{},\PYZsq{}}
        \PY{n}{out\PYZus{}seq} \PY{o}{=} \PY{p}{[}\PY{p}{]}                                                   \PY{c+c1}{\PYZsh{} create empty array  }
        
        \PY{k}{for} \PY{n}{j} \PY{o+ow}{in} \PY{n}{bi\PYZus{}num}\PY{p}{:}                                               \PY{c+c1}{\PYZsh{} iterate through total number of input}
            \PY{n}{i} \PY{o}{=} \PY{n+nb}{int}\PY{p}{(}\PY{n}{j}\PY{p}{,}\PY{l+m+mi}{2}\PY{p}{)} 
            \PY{k}{if} \PY{n}{i}\PY{o}{\PYZpc{}}\PY{k}{5} == 0:                                               \PYZsh{} check if divisible by 5  
                \PY{n}{out\PYZus{}seq}\PY{o}{.}\PY{n}{append}\PY{p}{(}\PY{n}{j}\PY{p}{)}                                      \PY{c+c1}{\PYZsh{} append if divisible by 5  }
        \PY{n+nb}{print}\PY{p}{(}\PY{l+s+s1}{\PYZsq{}}\PY{l+s+s1}{,}\PY{l+s+s1}{\PYZsq{}}\PY{o}{.}\PY{n}{join}\PY{p}{(}\PY{n}{out\PYZus{}seq}\PY{p}{)}\PY{p}{)}                                       \PY{c+c1}{\PYZsh{} join the numbers and display}
\end{Verbatim}


    \begin{Verbatim}[commandchars=\\\{\}]
0100,0011,1010,1001
1010

    \end{Verbatim}

    \subsubsection{Version 2}\label{version-2}

    \begin{Verbatim}[commandchars=\\\{\}]
{\color{incolor}In [{\color{incolor}10}]:} \PY{n}{out\PYZus{}value} \PY{o}{=} \PY{p}{[}\PY{p}{]}                                                  \PY{c+c1}{\PYZsh{} create empty array}
         \PY{n}{items}\PY{o}{=}\PY{p}{[}\PY{n}{i} \PY{k}{for} \PY{n}{i} \PY{o+ow}{in} \PY{n+nb}{input}\PY{p}{(}\PY{p}{)}\PY{o}{.}\PY{n}{split}\PY{p}{(}\PY{l+s+s1}{\PYZsq{}}\PY{l+s+s1}{,}\PY{l+s+s1}{\PYZsq{}}\PY{p}{)}\PY{p}{]}                           \PY{c+c1}{\PYZsh{} input sequence and split at \PYZsq{},\PYZsq{}}
         
         \PY{k}{for} \PY{n}{j} \PY{o+ow}{in} \PY{n}{items}\PY{p}{:}                                                 \PY{c+c1}{\PYZsh{} iterate through total number of input}
             \PY{n}{i} \PY{o}{=} \PY{n+nb}{int}\PY{p}{(}\PY{n}{j}\PY{p}{,} \PY{l+m+mi}{2}\PY{p}{)}
             \PY{k}{if} \PY{o+ow}{not} \PY{n}{i}\PY{o}{\PYZpc{}}\PY{k}{5}:                                                 \PYZsh{} check if divisible by 5
                 \PY{n}{out\PYZus{}value}\PY{o}{.}\PY{n}{append}\PY{p}{(}\PY{n}{j}\PY{p}{)}                                     \PY{c+c1}{\PYZsh{} append if divisible by 5}
         
         \PY{n+nb}{print}\PY{p}{(}\PY{l+s+s1}{\PYZsq{}}\PY{l+s+s1}{,}\PY{l+s+s1}{\PYZsq{}}\PY{o}{.}\PY{n}{join}\PY{p}{(}\PY{n}{out\PYZus{}value}\PY{p}{)}\PY{p}{)}                                      \PY{c+c1}{\PYZsh{} join the numbers and display }
\end{Verbatim}


    \begin{Verbatim}[commandchars=\\\{\}]
0100,0011,1010,1001
1010

    \end{Verbatim}

    \subsection{Question 6}\label{question-6}

    Write a program, which will find all such numbers between 1000 and 3000
(both included) such that each digit of the number is an even number.

The numbers obtained should be printed in a comma-separated sequence on
a single line.

    \subsubsection{Version 1}\label{version-1}

    \begin{Verbatim}[commandchars=\\\{\}]
{\color{incolor}In [{\color{incolor}52}]:} \PY{k}{def} \PY{n+nf}{even}\PY{p}{(}\PY{n}{a}\PY{p}{,}\PY{n}{z}\PY{p}{)}\PY{p}{:}                                             \PY{c+c1}{\PYZsh{} define function even accepting start and end number }
             
             \PY{n}{sequence} \PY{o}{=} \PY{p}{[}\PY{p}{]}                                          \PY{c+c1}{\PYZsh{} create empty array  }
             
             \PY{k}{for} \PY{n}{num} \PY{o+ow}{in} \PY{n+nb}{range}\PY{p}{(}\PY{n}{a}\PY{p}{,}\PY{n}{z}\PY{p}{)}\PY{p}{:}                                 \PY{c+c1}{\PYZsh{} iterate through the total number }
                 \PY{n}{is\PYZus{}even} \PY{o}{=} \PY{k+kc}{True}                                     \PY{c+c1}{\PYZsh{} default as true}
                 \PY{n}{total} \PY{o}{=} \PY{n+nb}{str}\PY{p}{(}\PY{n}{num}\PY{p}{)}                                   \PY{c+c1}{\PYZsh{} make the total number as str}
                 
                 \PY{k}{for} \PY{n}{i} \PY{o+ow}{in} \PY{n}{total}\PY{p}{:}                                    \PY{c+c1}{\PYZsh{} iterate through total number}
                     \PY{k}{if} \PY{n+nb}{int}\PY{p}{(}\PY{n}{i}\PY{p}{)} \PY{o}{\PYZpc{}} \PY{l+m+mi}{2} \PY{o}{!=} \PY{l+m+mi}{0}\PY{p}{:}                            \PY{c+c1}{\PYZsh{} check if every digit is divisible by 2 }
                         \PY{n}{is\PYZus{}even} \PY{o}{=} \PY{k+kc}{False}
                 \PY{k}{if} \PY{n}{is\PYZus{}even} \PY{o}{==} \PY{k+kc}{True}\PY{p}{:}
                     \PY{n}{sequence}\PY{o}{.}\PY{n}{append}\PY{p}{(}\PY{n+nb}{str}\PY{p}{(}\PY{n}{num}\PY{p}{)}\PY{p}{)}                      \PY{c+c1}{\PYZsh{} append if every number is divisible by 2}
             \PY{k}{return}\PY{p}{(}\PY{n}{sequence}\PY{p}{)}                                       \PY{c+c1}{\PYZsh{} return the array }
\end{Verbatim}


    \begin{Verbatim}[commandchars=\\\{\}]
{\color{incolor}In [{\color{incolor}53}]:} \PY{n+nb}{print}\PY{p}{(}\PY{l+s+s2}{\PYZdq{}}\PY{l+s+s2}{the sequence is : }\PY{l+s+se}{\PYZbs{}n}\PY{l+s+se}{\PYZbs{}n}\PY{l+s+s2}{ }\PY{l+s+si}{\PYZob{}\PYZcb{}}\PY{l+s+s2}{ }\PY{l+s+s2}{\PYZdq{}}\PY{o}{.}\PY{n}{format}\PY{p}{(}\PY{l+s+s2}{\PYZdq{}}\PY{l+s+s2}{,}\PY{l+s+s2}{\PYZdq{}}\PY{o}{.}\PY{n}{join}\PY{p}{(}\PY{n}{even}\PY{p}{(}\PY{l+m+mi}{1000}\PY{p}{,}\PY{l+m+mi}{3001}\PY{p}{)}\PY{p}{)}\PY{p}{)}\PY{p}{)}     \PY{c+c1}{\PYZsh{} display array after passing through function}
\end{Verbatim}


    \begin{Verbatim}[commandchars=\\\{\}]
the sequence is : 

 2000,2002,2004,2006,2008,2020,2022,2024,2026,2028,2040,2042,2044,2046,2048,2060,2062,2064,2066,2068,2080,2082,2084,2086,2088,2200,2202,2204,2206,2208,2220,2222,2224,2226,2228,2240,2242,2244,2246,2248,2260,2262,2264,2266,2268,2280,2282,2284,2286,2288,2400,2402,2404,2406,2408,2420,2422,2424,2426,2428,2440,2442,2444,2446,2448,2460,2462,2464,2466,2468,2480,2482,2484,2486,2488,2600,2602,2604,2606,2608,2620,2622,2624,2626,2628,2640,2642,2644,2646,2648,2660,2662,2664,2666,2668,2680,2682,2684,2686,2688,2800,2802,2804,2806,2808,2820,2822,2824,2826,2828,2840,2842,2844,2846,2848,2860,2862,2864,2866,2868,2880,2882,2884,2886,2888 

    \end{Verbatim}

    \subsubsection{Version 2}\label{version-2}

    \begin{Verbatim}[commandchars=\\\{\}]
{\color{incolor}In [{\color{incolor}56}]:} \PY{n}{seq} \PY{o}{=} \PY{p}{[}\PY{p}{]}                                                     \PY{c+c1}{\PYZsh{} create empty array }
         
         \PY{k}{for} \PY{n}{i} \PY{o+ow}{in} \PY{n+nb}{range}\PY{p}{(}\PY{l+m+mi}{1000}\PY{p}{,} \PY{l+m+mi}{3001}\PY{p}{)}\PY{p}{:}                                  \PY{c+c1}{\PYZsh{} iterate through the total number}
             \PY{n}{ev} \PY{o}{=} \PY{n+nb}{str}\PY{p}{(}\PY{n}{i}\PY{p}{)}
             \PY{k}{if} \PY{p}{(}\PY{n+nb}{int}\PY{p}{(}\PY{n}{ev}\PY{p}{[}\PY{l+m+mi}{0}\PY{p}{]}\PY{p}{)}\PY{o}{\PYZpc{}}\PY{k}{2}==0) and (int(ev[1])\PYZpc{}2==0) and (int(ev[2])\PYZpc{}2==0) and (int(ev[3])\PYZpc{}2==0):          \PYZsh{} check if every digit is divisible by 2
                 \PY{n}{seq}\PY{o}{.}\PY{n}{append}\PY{p}{(}\PY{n}{ev}\PY{p}{)}                                       \PY{c+c1}{\PYZsh{} append if every number is divisible by 2 }
                 
         \PY{n+nb}{print}\PY{p}{(}\PY{l+s+s2}{\PYZdq{}}\PY{l+s+s2}{the sequence is : }\PY{l+s+se}{\PYZbs{}n}\PY{l+s+se}{\PYZbs{}n}\PY{l+s+s2}{ }\PY{l+s+s2}{\PYZdq{}}\PY{p}{,}\PY{l+s+s2}{\PYZdq{}}\PY{l+s+s2}{,}\PY{l+s+s2}{\PYZdq{}}\PY{o}{.}\PY{n}{join}\PY{p}{(}\PY{n}{seq}\PY{p}{)}\PY{p}{)}               \PY{c+c1}{\PYZsh{} return the array  }
\end{Verbatim}


    \begin{Verbatim}[commandchars=\\\{\}]
the sequence is : 

  2000,2002,2004,2006,2008,2020,2022,2024,2026,2028,2040,2042,2044,2046,2048,2060,2062,2064,2066,2068,2080,2082,2084,2086,2088,2200,2202,2204,2206,2208,2220,2222,2224,2226,2228,2240,2242,2244,2246,2248,2260,2262,2264,2266,2268,2280,2282,2284,2286,2288,2400,2402,2404,2406,2408,2420,2422,2424,2426,2428,2440,2442,2444,2446,2448,2460,2462,2464,2466,2468,2480,2482,2484,2486,2488,2600,2602,2604,2606,2608,2620,2622,2624,2626,2628,2640,2642,2644,2646,2648,2660,2662,2664,2666,2668,2680,2682,2684,2686,2688,2800,2802,2804,2806,2808,2820,2822,2824,2826,2828,2840,2842,2844,2846,2848,2860,2862,2864,2866,2868,2880,2882,2884,2886,2888

    \end{Verbatim}

    \subsection{Question 7}\label{question-7}

    Write a program that accepts a sentence and calculate the number of
letters and digits.

Suppose the following input is supplied to the program:

hello world! 123

Then, the output should be:

LETTERS 10

DIGITS 3

    \subsubsection{Version 1}\label{version-1}

    \begin{Verbatim}[commandchars=\\\{\}]
{\color{incolor}In [{\color{incolor}9}]:} \PY{n}{string\PYZus{}in} \PY{o}{=} \PY{n+nb}{input}\PY{p}{(}\PY{l+s+s2}{\PYZdq{}}\PY{l+s+s2}{ Input a string : }\PY{l+s+s2}{\PYZdq{}}\PY{p}{)}                          \PY{c+c1}{\PYZsh{} input sentence}
        
        \PY{n}{digit} \PY{o}{=} \PY{l+m+mi}{0}                                                        \PY{c+c1}{\PYZsh{} assign digit variable to 0}
        \PY{n}{letter} \PY{o}{=} \PY{l+m+mi}{0}                                                       \PY{c+c1}{\PYZsh{} assign letter variable to 0}
        
        \PY{k}{for} \PY{n}{i} \PY{o+ow}{in} \PY{n}{string\PYZus{}in}\PY{p}{:}                                              \PY{c+c1}{\PYZsh{} iterate total numeber of charecters }
            \PY{k}{if} \PY{n}{i}\PY{o}{.}\PY{n}{isalpha}\PY{p}{(}\PY{p}{)}\PY{p}{:}                                              \PY{c+c1}{\PYZsh{} check if charecter is an alphabet}
                \PY{n}{letter} \PY{o}{+}\PY{o}{=} \PY{l+m+mi}{1}                                              \PY{c+c1}{\PYZsh{} increase letter count by 1}
            \PY{k}{elif} \PY{n}{i}\PY{o}{.}\PY{n}{isdigit}\PY{p}{(}\PY{p}{)}\PY{p}{:}                                            \PY{c+c1}{\PYZsh{} check if charecter is an digit}
                \PY{n}{digit} \PY{o}{+}\PY{o}{=} \PY{l+m+mi}{1}                                               \PY{c+c1}{\PYZsh{} increase digit count by 1 }
            \PY{k}{else}\PY{p}{:}
                \PY{k}{pass}
        
        \PY{n+nb}{print}\PY{p}{(}\PY{l+s+s2}{\PYZdq{}}\PY{l+s+s2}{The number of letters are : }\PY{l+s+si}{\PYZob{}\PYZcb{}}\PY{l+s+s2}{\PYZdq{}}\PY{o}{.}\PY{n}{format}\PY{p}{(}\PY{n}{letter}\PY{p}{)}\PY{p}{)}           \PY{c+c1}{\PYZsh{} display total number of letter}
        \PY{n+nb}{print}\PY{p}{(}\PY{l+s+s2}{\PYZdq{}}\PY{l+s+s2}{The number of digits are  : }\PY{l+s+si}{\PYZob{}\PYZcb{}}\PY{l+s+s2}{\PYZdq{}}\PY{o}{.}\PY{n}{format}\PY{p}{(}\PY{n}{digit}\PY{p}{)}\PY{p}{)}            \PY{c+c1}{\PYZsh{} display taoal number of digit}
\end{Verbatim}


    \begin{Verbatim}[commandchars=\\\{\}]
 Input a string : hello world! 123
The number of letters are : 10
The number of digits are  : 3

    \end{Verbatim}

    \subsubsection{Version 2}\label{version-2}

    \begin{Verbatim}[commandchars=\\\{\}]
{\color{incolor}In [{\color{incolor}10}]:} \PY{n}{string\PYZus{}in} \PY{o}{=} \PY{n+nb}{input}\PY{p}{(}\PY{l+s+s2}{\PYZdq{}}\PY{l+s+s2}{ Input a string : }\PY{l+s+s2}{\PYZdq{}}\PY{p}{)}                          \PY{c+c1}{\PYZsh{} input sentence}
         
         \PY{n}{d} \PY{o}{=} \PY{p}{\PYZob{}}\PY{l+s+s2}{\PYZdq{}}\PY{l+s+s2}{DIGITS}\PY{l+s+s2}{\PYZdq{}}\PY{p}{:}\PY{l+m+mi}{0}\PY{p}{,} \PY{l+s+s2}{\PYZdq{}}\PY{l+s+s2}{LETTERS}\PY{l+s+s2}{\PYZdq{}}\PY{p}{:}\PY{l+m+mi}{0}\PY{p}{\PYZcb{}}                                    \PY{c+c1}{\PYZsh{} assign digit and letter variable to 0}
         \PY{k}{for} \PY{n}{i} \PY{o+ow}{in} \PY{n}{string\PYZus{}in}\PY{p}{:}                                              \PY{c+c1}{\PYZsh{} iterate total numeber of charecters  }
             \PY{k}{if} \PY{n}{i}\PY{o}{.}\PY{n}{isdigit}\PY{p}{(}\PY{p}{)}\PY{p}{:}                                              \PY{c+c1}{\PYZsh{} check if charecter is an digit}
                 \PY{n}{d}\PY{p}{[}\PY{l+s+s2}{\PYZdq{}}\PY{l+s+s2}{DIGITS}\PY{l+s+s2}{\PYZdq{}}\PY{p}{]}\PY{o}{+}\PY{o}{=}\PY{l+m+mi}{1}                                           \PY{c+c1}{\PYZsh{} increase digit count by 1  }
             \PY{k}{elif} \PY{n}{i}\PY{o}{.}\PY{n}{isalpha}\PY{p}{(}\PY{p}{)}\PY{p}{:}                                            \PY{c+c1}{\PYZsh{} check if charecter is an alphabet}
                 \PY{n}{d}\PY{p}{[}\PY{l+s+s2}{\PYZdq{}}\PY{l+s+s2}{LETTERS}\PY{l+s+s2}{\PYZdq{}}\PY{p}{]}\PY{o}{+}\PY{o}{=}\PY{l+m+mi}{1}                                          \PY{c+c1}{\PYZsh{} increase letter count by 1}
             \PY{k}{else}\PY{p}{:}
                 \PY{k}{pass}
         \PY{n+nb}{print}\PY{p}{(}\PY{l+s+s2}{\PYZdq{}}\PY{l+s+s2}{LETTERS}\PY{l+s+s2}{\PYZdq{}}\PY{p}{,} \PY{n}{d}\PY{p}{[}\PY{l+s+s2}{\PYZdq{}}\PY{l+s+s2}{LETTERS}\PY{l+s+s2}{\PYZdq{}}\PY{p}{]}\PY{p}{)}                                   \PY{c+c1}{\PYZsh{} display total number of letter}
         \PY{n+nb}{print}\PY{p}{(}\PY{l+s+s2}{\PYZdq{}}\PY{l+s+s2}{DIGITS}\PY{l+s+s2}{\PYZdq{}}\PY{p}{,} \PY{n}{d}\PY{p}{[}\PY{l+s+s2}{\PYZdq{}}\PY{l+s+s2}{DIGITS}\PY{l+s+s2}{\PYZdq{}}\PY{p}{]}\PY{p}{)}                                     \PY{c+c1}{\PYZsh{} display taoal number of digit}
\end{Verbatim}


    \begin{Verbatim}[commandchars=\\\{\}]
 Input a string : hello world! 123
LETTERS 10
DIGITS 3

    \end{Verbatim}

    \subsection{Question 8}\label{question-8}

    Write a program that computes the value of a+aa+aaa+aaaa with a given
digit as the value of a.

Suppose the following input is supplied to the program:

9

Then, the output should be:

11106

    \subsubsection{Version 1}\label{version-1}

    \begin{Verbatim}[commandchars=\\\{\}]
{\color{incolor}In [{\color{incolor}17}]:} \PY{n}{number} \PY{o}{=} \PY{n+nb}{input}\PY{p}{(}\PY{l+s+s2}{\PYZdq{}}\PY{l+s+s2}{enter the digit : }\PY{l+s+s2}{\PYZdq{}}\PY{p}{)}                             \PY{c+c1}{\PYZsh{} input digit}
         
         \PY{n}{p1} \PY{o}{=} \PY{n+nb}{int}\PY{p}{(} \PY{l+s+s2}{\PYZdq{}}\PY{l+s+si}{\PYZpc{}s}\PY{l+s+s2}{\PYZdq{}} \PY{o}{\PYZpc{}} \PY{n}{number} \PY{p}{)}                                        
         \PY{n}{p2} \PY{o}{=} \PY{n+nb}{int}\PY{p}{(} \PY{l+s+s2}{\PYZdq{}}\PY{l+s+si}{\PYZpc{}s}\PY{l+s+si}{\PYZpc{}s}\PY{l+s+s2}{\PYZdq{}} \PY{o}{\PYZpc{}} \PY{p}{(}\PY{n}{number}\PY{p}{,}\PY{n}{number}\PY{p}{)} \PY{p}{)}
         \PY{n}{p3} \PY{o}{=} \PY{n+nb}{int}\PY{p}{(} \PY{l+s+s2}{\PYZdq{}}\PY{l+s+si}{\PYZpc{}s}\PY{l+s+si}{\PYZpc{}s}\PY{l+s+si}{\PYZpc{}s}\PY{l+s+s2}{\PYZdq{}} \PY{o}{\PYZpc{}} \PY{p}{(}\PY{n}{number}\PY{p}{,}\PY{n}{number}\PY{p}{,}\PY{n}{number}\PY{p}{)} \PY{p}{)}
         \PY{n}{p4} \PY{o}{=} \PY{n+nb}{int}\PY{p}{(} \PY{l+s+s2}{\PYZdq{}}\PY{l+s+si}{\PYZpc{}s}\PY{l+s+si}{\PYZpc{}s}\PY{l+s+si}{\PYZpc{}s}\PY{l+s+si}{\PYZpc{}s}\PY{l+s+s2}{\PYZdq{}} \PY{o}{\PYZpc{}} \PY{p}{(}\PY{n}{number}\PY{p}{,}\PY{n}{number}\PY{p}{,}\PY{n}{number}\PY{p}{,}\PY{n}{number}\PY{p}{)} \PY{p}{)}
         
         \PY{n}{sum\PYZus{}of\PYZus{}all} \PY{o}{=} \PY{n}{p1} \PY{o}{+} \PY{n}{p2} \PY{o}{+} \PY{n}{p3} \PY{o}{+} \PY{n}{p4}                                   \PY{c+c1}{\PYZsh{} add all value }
         
         \PY{n+nb}{print}\PY{p}{(}\PY{l+s+s2}{\PYZdq{}}\PY{l+s+s2}{the sum value of digti is : }\PY{l+s+si}{\PYZob{}\PYZcb{}}\PY{l+s+s2}{ }\PY{l+s+s2}{\PYZdq{}}\PY{o}{.}\PY{n}{format}\PY{p}{(}\PY{n}{sum\PYZus{}of\PYZus{}all}\PY{p}{)}\PY{p}{)}      \PY{c+c1}{\PYZsh{} display result}
\end{Verbatim}


    \begin{Verbatim}[commandchars=\\\{\}]
enter the digit : 9
the sum value of digti is : 11106 

    \end{Verbatim}

    \subsubsection{Version 2}\label{version-2}

    \begin{Verbatim}[commandchars=\\\{\}]
{\color{incolor}In [{\color{incolor}24}]:} \PY{n}{number} \PY{o}{=} \PY{n+nb}{input}\PY{p}{(}\PY{l+s+s2}{\PYZdq{}}\PY{l+s+s2}{enter number :}\PY{l+s+s2}{\PYZdq{}}\PY{p}{)}                                 \PY{c+c1}{\PYZsh{} input digit}
         \PY{n}{DIGIT} \PY{o}{=} \PY{l+m+mi}{4}
         \PY{n}{res} \PY{o}{=} \PY{l+m+mi}{0}
         \PY{k}{for} \PY{n}{i} \PY{o+ow}{in} \PY{n+nb}{range}\PY{p}{(}\PY{l+m+mi}{1}\PY{p}{,}\PY{n}{DIGIT}\PY{o}{+}\PY{l+m+mi}{1}\PY{p}{)}\PY{p}{:}                                       \PY{c+c1}{\PYZsh{} add all value}
             \PY{n}{res} \PY{o}{+}\PY{o}{=} \PY{n+nb}{int}\PY{p}{(}\PY{n+nb}{str}\PY{p}{(}\PY{n}{number}\PY{p}{)}\PY{o}{*}\PY{n}{i}\PY{p}{)}                                    \PY{c+c1}{\PYZsh{} mult and add }
         \PY{n+nb}{print}\PY{p}{(}\PY{l+s+s2}{\PYZdq{}}\PY{l+s+s2}{the sum value of digti is : }\PY{l+s+si}{\PYZob{}\PYZcb{}}\PY{l+s+s2}{ }\PY{l+s+s2}{\PYZdq{}}\PY{o}{.}\PY{n}{format}\PY{p}{(}\PY{n}{res}\PY{p}{)}\PY{p}{)}             \PY{c+c1}{\PYZsh{} display result}
\end{Verbatim}


    \begin{Verbatim}[commandchars=\\\{\}]
enter number :9
the sum value of digti is : 11106 

    \end{Verbatim}

    \subsection{Question 9}\label{question-9}

    Write a program that computes the net amount of a bank account based a
transaction log from console input. The transaction log format is shown
as following:

D 100

W 200

...

D means deposit while W means withdrawal.

Suppose the following input is supplied to the program:

D 300

D 300

W 200

D 100

Then, the output should be:

500

    \subsubsection{Version 1}\label{version-1}

    \begin{Verbatim}[commandchars=\\\{\}]
{\color{incolor}In [{\color{incolor}17}]:} \PY{n}{net\PYZus{}transaction} \PY{o}{=} \PY{l+m+mi}{0}                                                \PY{c+c1}{\PYZsh{} set initial amount to 0}
         
         \PY{k}{while} \PY{k+kc}{True}\PY{p}{:}                                                        \PY{c+c1}{\PYZsh{} keep checking for input }
             \PY{n}{input\PYZus{}trans} \PY{o}{=} \PY{n+nb}{input}\PY{p}{(}\PY{l+s+s2}{\PYZdq{}}\PY{l+s+s2}{\PYZhy{}\PYZhy{}\PYZgt{} }\PY{l+s+s2}{\PYZdq{}}\PY{p}{)}\PY{o}{.}\PY{n}{split}\PY{p}{(}\PY{p}{)}                            \PY{c+c1}{\PYZsh{} input value and split      }
             \PY{k}{if} \PY{o+ow}{not} \PY{n}{input\PYZus{}trans}\PY{p}{:}
                 \PY{k}{break}\PY{p}{;}                                                     \PY{c+c1}{\PYZsh{} break if there is no input}
         
             \PY{n}{net\PYZus{}amount} \PY{o}{=} \PY{n+nb}{int}\PY{p}{(}\PY{n}{input\PYZus{}trans}\PY{p}{[}\PY{l+m+mi}{1}\PY{p}{]}\PY{p}{)}                               
             \PY{k}{if} \PY{n}{input\PYZus{}trans}\PY{p}{[}\PY{l+m+mi}{0}\PY{p}{]} \PY{o}{==} \PY{l+s+s1}{\PYZsq{}}\PY{l+s+s1}{D}\PY{l+s+s1}{\PYZsq{}}\PY{p}{:}                                      \PY{c+c1}{\PYZsh{} check if the initial char starts with D for Deposit}
                 \PY{n}{net\PYZus{}transaction} \PY{o}{+}\PY{o}{=} \PY{n}{net\PYZus{}amount}                              \PY{c+c1}{\PYZsh{} add it to the net amount}
             \PY{k}{elif} \PY{n}{input\PYZus{}trans}\PY{p}{[}\PY{l+m+mi}{0}\PY{p}{]} \PY{o}{==} \PY{l+s+s1}{\PYZsq{}}\PY{l+s+s1}{W}\PY{l+s+s1}{\PYZsq{}}\PY{p}{:}                                    \PY{c+c1}{\PYZsh{} check if the initial char starts with W for Withdrawal}
                 \PY{n}{net\PYZus{}transaction} \PY{o}{\PYZhy{}}\PY{o}{=} \PY{n}{net\PYZus{}amount}                              \PY{c+c1}{\PYZsh{} deduct it from the net amount }
         
         \PY{n+nb}{print}\PY{p}{(}\PY{l+s+s2}{\PYZdq{}}\PY{l+s+s2}{The net amount is : }\PY{l+s+si}{\PYZob{}\PYZcb{}}\PY{l+s+s2}{ }\PY{l+s+s2}{\PYZdq{}}\PY{o}{.}\PY{n}{format}\PY{p}{(}\PY{n}{net\PYZus{}transaction}\PY{p}{)}\PY{p}{)}           \PY{c+c1}{\PYZsh{} display the remaining amount}
\end{Verbatim}


    \begin{Verbatim}[commandchars=\\\{\}]
--> D 300
--> D 300
--> W 200
--> D 100
--> 
The net amount is : 500 

    \end{Verbatim}

    \subsubsection{Version 2}\label{version-2}

    \begin{Verbatim}[commandchars=\\\{\}]
{\color{incolor}In [{\color{incolor}23}]:} \PY{n}{net\PYZus{}amount} \PY{o}{=} \PY{l+m+mi}{0}                                                      \PY{c+c1}{\PYZsh{} set initial amount to 0  }
         
         \PY{k}{while} \PY{k+kc}{True}\PY{p}{:}                                                         \PY{c+c1}{\PYZsh{} keep checking for input}
             \PY{n}{s} \PY{o}{=} \PY{n+nb}{input}\PY{p}{(}\PY{l+s+s2}{\PYZdq{}}\PY{l+s+s2}{\PYZhy{}\PYZhy{}\PYZgt{} }\PY{l+s+s2}{\PYZdq{}}\PY{p}{)}                                               \PY{c+c1}{\PYZsh{} input value and split}
             \PY{k}{if} \PY{o+ow}{not} \PY{n}{s}\PY{p}{:}
                 \PY{k}{break}                                                       \PY{c+c1}{\PYZsh{} break if there is no input}
             \PY{n}{values} \PY{o}{=} \PY{n}{s}\PY{o}{.}\PY{n}{split}\PY{p}{(}\PY{l+s+s2}{\PYZdq{}}\PY{l+s+s2}{ }\PY{l+s+s2}{\PYZdq{}}\PY{p}{)}                                            
             \PY{n}{operation} \PY{o}{=} \PY{n}{values}\PY{p}{[}\PY{l+m+mi}{0}\PY{p}{]}
             \PY{n}{amount} \PY{o}{=} \PY{n+nb}{int}\PY{p}{(}\PY{n}{values}\PY{p}{[}\PY{l+m+mi}{1}\PY{p}{]}\PY{p}{)}
             \PY{k}{if} \PY{n}{operation}\PY{o}{==}\PY{l+s+s2}{\PYZdq{}}\PY{l+s+s2}{D}\PY{l+s+s2}{\PYZdq{}}\PY{p}{:}                                              \PY{c+c1}{\PYZsh{} check if the initial char starts with D for Deposit}
                 \PY{n}{net\PYZus{}amount}\PY{o}{+}\PY{o}{=}\PY{n}{amount}                                          \PY{c+c1}{\PYZsh{} add it to the net amount}
             \PY{k}{elif} \PY{n}{operation}\PY{o}{==}\PY{l+s+s2}{\PYZdq{}}\PY{l+s+s2}{W}\PY{l+s+s2}{\PYZdq{}}\PY{p}{:}                                            \PY{c+c1}{\PYZsh{} check if the initial char starts with W for Withdrawal}
                 \PY{n}{net\PYZus{}amount}\PY{o}{\PYZhy{}}\PY{o}{=}\PY{n}{amount}                                          \PY{c+c1}{\PYZsh{} deduct it from the net amount          }
             \PY{k}{else}\PY{p}{:}
                 \PY{k}{pass}
         \PY{n+nb}{print}\PY{p}{(}\PY{l+s+s2}{\PYZdq{}}\PY{l+s+s2}{The net amount is : }\PY{l+s+si}{\PYZob{}\PYZcb{}}\PY{l+s+s2}{ }\PY{l+s+s2}{\PYZdq{}}\PY{o}{.}\PY{n}{format}\PY{p}{(}\PY{n}{net\PYZus{}amount}\PY{p}{)}\PY{p}{)}                 \PY{c+c1}{\PYZsh{} display the remaining amount }
\end{Verbatim}


    \begin{Verbatim}[commandchars=\\\{\}]
--> D 300
--> D 300
--> W 200
--> D 100
--> 
The net amount is : 500 

    \end{Verbatim}

    \subsection{Question 10}\label{question-10}

    Write a program to compute:

f(n)=f(n-1)+100 when n\textgreater{}0

and f(0)=1

with a given n input by console (n\textgreater{}0).

    \subsubsection{version 1}\label{version-1}

    \begin{Verbatim}[commandchars=\\\{\}]
{\color{incolor}In [{\color{incolor}8}]:} \PY{k}{def} \PY{n+nf}{f}\PY{p}{(}\PY{n}{n}\PY{p}{)}\PY{p}{:}                                                         \PY{c+c1}{\PYZsh{} define function to accept a digit}
            \PY{k}{if} \PY{n}{n} \PY{o}{==} \PY{l+m+mi}{0}\PY{p}{:}
                \PY{k}{return} \PY{l+m+mi}{1}                                                  \PY{c+c1}{\PYZsh{} return 1 if number = 0 }
            \PY{k}{else}\PY{p}{:}
                \PY{k}{return} \PY{n}{f}\PY{p}{(}\PY{n}{n}\PY{o}{\PYZhy{}}\PY{l+m+mi}{1}\PY{p}{)}\PY{o}{+}\PY{l+m+mi}{100}                                         \PY{c+c1}{\PYZsh{} perform recursive function and return value}
\end{Verbatim}


    \begin{Verbatim}[commandchars=\\\{\}]
{\color{incolor}In [{\color{incolor}38}]:} \PY{n}{input\PYZus{}n} \PY{o}{=} \PY{n+nb}{int}\PY{p}{(}\PY{n+nb}{input}\PY{p}{(}\PY{l+s+s2}{\PYZdq{}}\PY{l+s+s2}{enter number : }\PY{l+s+s2}{\PYZdq{}}\PY{p}{)}\PY{p}{)}                           \PY{c+c1}{\PYZsh{} input digit/value}
         \PY{n+nb}{print}\PY{p}{(}\PY{l+s+s2}{\PYZdq{}}\PY{l+s+s2}{f(}\PY{l+s+si}{\PYZob{}0\PYZcb{}}\PY{l+s+s2}{) = }\PY{l+s+si}{\PYZob{}1\PYZcb{}}\PY{l+s+s2}{\PYZdq{}}\PY{o}{.}\PY{n}{format}\PY{p}{(}\PY{p}{(}\PY{n}{input\PYZus{}n}\PY{p}{)}\PY{p}{,} \PY{n}{f}\PY{p}{(}\PY{n}{input\PYZus{}n}\PY{p}{)}\PY{p}{)}\PY{p}{)}               \PY{c+c1}{\PYZsh{} display the computed value}
\end{Verbatim}


    \begin{Verbatim}[commandchars=\\\{\}]
enter number : 3
f(3) = 301

    \end{Verbatim}

    \begin{Verbatim}[commandchars=\\\{\}]
{\color{incolor}In [{\color{incolor}37}]:} \PY{k}{def} \PY{n+nf}{f2}\PY{p}{(}\PY{n}{n}\PY{p}{)}\PY{p}{:}
             \PY{n}{ans} \PY{o}{=} \PY{l+m+mi}{0}
             \PY{k}{if} \PY{n}{n} \PY{o}{==} \PY{l+m+mi}{0}\PY{p}{:}
                 \PY{k}{return} \PY{l+m+mi}{1}
             \PY{k}{else}\PY{p}{:}
                 \PY{k}{return} \PY{n}{f2}\PY{p}{(}\PY{n}{n}\PY{o}{\PYZhy{}}\PY{l+m+mi}{1}\PY{p}{)}\PY{o}{+}\PY{l+m+mi}{100}
         
         
         \PY{n}{input\PYZus{}n} \PY{o}{=} \PY{n+nb}{int}\PY{p}{(}\PY{n+nb}{input}\PY{p}{(}\PY{l+s+s2}{\PYZdq{}}\PY{l+s+s2}{enter number : }\PY{l+s+s2}{\PYZdq{}}\PY{p}{)}\PY{p}{)}
         
         \PY{n+nb}{print}\PY{p}{(}\PY{l+s+s2}{\PYZdq{}}\PY{l+s+s2}{f(}\PY{l+s+si}{\PYZob{}0\PYZcb{}}\PY{l+s+s2}{) = }\PY{l+s+si}{\PYZob{}1\PYZcb{}}\PY{l+s+s2}{\PYZdq{}}\PY{o}{.}\PY{n}{format}\PY{p}{(}\PY{p}{(}\PY{n}{input\PYZus{}n}\PY{p}{)}\PY{p}{,} \PY{n}{f2}\PY{p}{(}\PY{n}{input\PYZus{}n}\PY{p}{)}\PY{p}{)}\PY{p}{)}
\end{Verbatim}


    \begin{Verbatim}[commandchars=\\\{\}]
enter number : 3
f(3) = 301

    \end{Verbatim}

    \subsection{Question 11}\label{question-11}

    Segment the following short text into sentences and words:

s = u"""DTU course 02820 is taught by Mr. Bartlomiej Wilkowski, Mr.
Marcin Marek Szewczyk \& Finn Arup Nielsen, Ph.D. Some of aspects of the
course are: machine learning and web 2.0. The telephone to Finn is (+45)
4525 3921, and his email is fn@imm.dtu.dk. A book published by O'Reilly
called 'Programming Collective Intelligence' might be useful. It costs
\$39.99 or 285.00 kroner in Polyteknisk Boghandle. Is 'Text Processing
in Python' appropriate for the course? Perhaps! The constructor function
in Python is called "\textbf{init}()". fMRI will not be a topic of the
course."""

Try both with the re module as well as with a function from nltk.

    \subsubsection{Version 1}\label{version-1}

    \begin{Verbatim}[commandchars=\\\{\}]
{\color{incolor}In [{\color{incolor}18}]:} \PY{k+kn}{import} \PY{n+nn}{nltk}                                                         \PY{c+c1}{\PYZsh{} import nltk library}
         
         \PY{n}{text} \PY{o}{=} \PY{n+nb}{input}\PY{p}{(}\PY{l+s+s2}{\PYZdq{}}\PY{l+s+s2}{Enter the sentence : }\PY{l+s+s2}{\PYZdq{}}\PY{p}{)}                               \PY{c+c1}{\PYZsh{} input the sentence }
         
         \PY{n}{sent\PYZus{}text} \PY{o}{=} \PY{n}{nltk}\PY{o}{.}\PY{n}{sent\PYZus{}tokenize}\PY{p}{(}\PY{n}{text}\PY{p}{)}                                \PY{c+c1}{\PYZsh{} this gives us a list of sentences}
         
         \PY{k}{for} \PY{n}{sentence} \PY{o+ow}{in} \PY{n}{sent\PYZus{}text}\PY{p}{:}                                          \PY{c+c1}{\PYZsh{} now loop over each sentence and tokenize it separately}
             \PY{n}{tokenized\PYZus{}text} \PY{o}{=} \PY{n}{nltk}\PY{o}{.}\PY{n}{word\PYZus{}tokenize}\PY{p}{(}\PY{n}{sentence}\PY{p}{)}
             \PY{n}{tagged} \PY{o}{=} \PY{n}{nltk}\PY{o}{.}\PY{n}{pos\PYZus{}tag}\PY{p}{(}\PY{n}{tokenized\PYZus{}text}\PY{p}{)}
             \PY{n+nb}{print}\PY{p}{(}\PY{n}{tagged}\PY{p}{)}                                                   \PY{c+c1}{\PYZsh{} print the parsed sentence}
\end{Verbatim}


    \begin{Verbatim}[commandchars=\\\{\}]
Enter the sentence : u"""DTU course 02820 is taught by Mr. Bartlomiej Wilkowski, Mr. Marcin Marek Szewczyk \& Finn Arup Nielsen, Ph.D. Some of aspects of the course are: machine learning and web 2.0. The telephone to Finn is (+45) 4525 3921, and his email is fn@imm.dtu.dk. A book published by O’Reilly called ’Programming Collective Intelligence’ might be useful. It costs \$39.99 or 285.00 kroner in Polyteknisk Boghandle. Is ’Text Processing in Python’ appropriate for the course? Perhaps! The constructor function in Python is called "\_\_init\_\_()". fMRI will not be a topic of the course."""
[('u', 'NN'), ("''", "''"), ("''", "''"), ("''", "''"), ('DTU', 'NNP'), ('course', 'NN'), ('02820', 'CD'), ('is', 'VBZ'), ('taught', 'VBN'), ('by', 'IN'), ('Mr.', 'NNP'), ('Bartlomiej', 'NNP'), ('Wilkowski', 'NNP'), (',', ','), ('Mr.', 'NNP'), ('Marcin', 'NNP'), ('Marek', 'NNP'), ('Szewczyk', 'NNP'), ('\&', 'CC'), ('Finn', 'NNP'), ('Arup', 'NNP'), ('Nielsen', 'NNP'), (',', ','), ('Ph.D', 'NNP'), ('.', '.')]
[('Some', 'DT'), ('of', 'IN'), ('aspects', 'NNS'), ('of', 'IN'), ('the', 'DT'), ('course', 'NN'), ('are', 'VBP'), (':', ':'), ('machine', 'NN'), ('learning', 'NN'), ('and', 'CC'), ('web', '\$'), ('2.0', 'CD'), ('.', '.')]
[('The', 'DT'), ('telephone', 'NN'), ('to', 'TO'), ('Finn', 'NNP'), ('is', 'VBZ'), ('(', '('), ('+45', 'NN'), (')', ')'), ('4525', 'CD'), ('3921', 'CD'), (',', ','), ('and', 'CC'), ('his', 'PRP\$'), ('email', 'NN'), ('is', 'VBZ'), ('fn', 'JJ'), ('@', 'NNP'), ('imm.dtu.dk', 'NN'), ('.', '.')]
[('A', 'DT'), ('book', 'NN'), ('published', 'VBN'), ('by', 'IN'), ('O', 'NNP'), ('’', 'NNP'), ('Reilly', 'NNP'), ('called', 'VBD'), ('’', 'RP'), ('Programming', 'NNP'), ('Collective', 'NNP'), ('Intelligence', 'NNP'), ('’', 'NNP'), ('might', 'MD'), ('be', 'VB'), ('useful', 'JJ'), ('.', '.')]
[('It', 'PRP'), ('costs', 'VBZ'), ('\$', '\$'), ('39.99', 'CD'), ('or', 'CC'), ('285.00', 'CD'), ('kroner', 'NN'), ('in', 'IN'), ('Polyteknisk', 'NNP'), ('Boghandle', 'NNP'), ('.', '.')]
[('Is', 'VBZ'), ('’', 'JJ'), ('Text', 'NNP'), ('Processing', 'NNP'), ('in', 'IN'), ('Python', 'NNP'), ('’', 'NNP'), ('appropriate', 'NN'), ('for', 'IN'), ('the', 'DT'), ('course', 'NN'), ('?', '.')]
[('Perhaps', 'RB'), ('!', '.')]
[('The', 'DT'), ('constructor', 'NN'), ('function', 'NN'), ('in', 'IN'), ('Python', 'NNP'), ('is', 'VBZ'), ('called', 'VBN'), ('``', '``'), ('\_\_init\_\_', 'NNP'), ('(', '('), (')', ')'), ("''", "''"), ('.', '.')]
[('fMRI', 'NN'), ('will', 'MD'), ('not', 'RB'), ('be', 'VB'), ('a', 'DT'), ('topic', 'NN'), ('of', 'IN'), ('the', 'DT'), ('course', 'NN'), ('.', '.'), ("''", "''"), ("''", "''"), ("''", "''")]

    \end{Verbatim}

    \subsubsection{Version 2}\label{version-2}

    \begin{Verbatim}[commandchars=\\\{\}]
{\color{incolor}In [{\color{incolor}19}]:} \PY{k+kn}{import} \PY{n+nn}{re}                                                             \PY{c+c1}{\PYZsh{} import re library}
         
         \PY{n}{sentence} \PY{o}{=} \PY{n+nb}{input}\PY{p}{(}\PY{l+s+s2}{\PYZdq{}}\PY{l+s+s2}{Enter the sentence : }\PY{l+s+s2}{\PYZdq{}}\PY{p}{)}                             \PY{c+c1}{\PYZsh{} input the sentence }
         \PY{n}{sentences} \PY{o}{=} \PY{n}{re}\PY{o}{.}\PY{n}{split}\PY{p}{(}\PY{l+s+sa}{r}\PY{l+s+s1}{\PYZsq{}}\PY{l+s+s1}{(?\PYZlt{}!}\PY{l+s+s1}{\PYZbs{}}\PY{l+s+s1}{w}\PY{l+s+s1}{\PYZbs{}}\PY{l+s+s1}{.}\PY{l+s+s1}{\PYZbs{}}\PY{l+s+s1}{w.)(?\PYZlt{}![A\PYZhy{}Z][a\PYZhy{}z][0\PYZhy{}9]}\PY{l+s+s1}{\PYZbs{}}\PY{l+s+s1}{.)(?\PYZlt{}=}\PY{l+s+s1}{\PYZbs{}}\PY{l+s+s1}{.|}\PY{l+s+s1}{\PYZbs{}}\PY{l+s+s1}{?)}\PY{l+s+s1}{\PYZbs{}}\PY{l+s+s1}{s}\PY{l+s+s1}{\PYZsq{}}\PY{p}{,} \PY{n}{sentence}\PY{p}{)}    \PY{c+c1}{\PYZsh{} split sentence using all charecter maekers and breakers}
         
         \PY{n+nb}{print}\PY{p}{(}\PY{l+s+s2}{\PYZdq{}}\PY{l+s+s2}{the parsed sentence is : }\PY{l+s+s2}{\PYZdq{}}\PY{p}{)}                                    \PY{c+c1}{\PYZsh{} iterate through all words and print the parsed sentence  }
         \PY{k}{for} \PY{n}{i} \PY{o+ow}{in} \PY{n}{sentences}\PY{p}{:}
                 \PY{n+nb}{print}\PY{p}{(}\PY{n}{i}\PY{p}{)}
\end{Verbatim}


    \begin{Verbatim}[commandchars=\\\{\}]
Enter the sentence : u"""DTU course 02820 is taught by Mr. Bartlomiej Wilkowski, Mr. Marcin Marek Szewczyk \& Finn Arup Nielsen, Ph.D. Some of aspects of the course are: machine learning and web 2.0. The telephone to Finn is (+45) 4525 3921, and his email is fn@imm.dtu.dk. A book published by O’Reilly called ’Programming Collective Intelligence’ might be useful. It costs \$39.99 or 285.00 kroner in Polyteknisk Boghandle. Is ’Text Processing in Python’ appropriate for the course? Perhaps! The constructor function in Python is called "\_\_init\_\_()". fMRI will not be a topic of the course."""
the parsed sentence is : 
u"""DTU course 02820 is taught by Mr.
Bartlomiej Wilkowski, Mr.
Marcin Marek Szewczyk \& Finn Arup Nielsen, Ph.D. Some of aspects of the course are: machine learning and web 2.0. The telephone to Finn is (+45) 4525 3921, and his email is fn@imm.dtu.dk.
A book published by O’Reilly called ’Programming Collective Intelligence’ might be useful.
It costs \$39.99 or 285.00 kroner in Polyteknisk Boghandle.
Is ’Text Processing in Python’ appropriate for the course?
Perhaps! The constructor function in Python is called "\_\_init\_\_()".
fMRI will not be a topic of the course."""

    \end{Verbatim}

    \begin{Verbatim}[commandchars=\\\{\}]
{\color{incolor}In [{\color{incolor}10}]:} \PY{c+c1}{\PYZsh{} from nltk.tokenize import TweetTokenizer, sent\PYZus{}tokenize}
         
         \PY{c+c1}{\PYZsh{} input\PYZus{}text = input()}
         
         \PY{c+c1}{\PYZsh{} tokenizer\PYZus{}words = TweetTokenizer()}
         \PY{c+c1}{\PYZsh{} tokens\PYZus{}sentences = [tokenizer\PYZus{}words.tokenize(t) for t in nltk.sent\PYZus{}tokenize(input\PYZus{}text)]}
         \PY{c+c1}{\PYZsh{} print(tokens\PYZus{}sentences)}
\end{Verbatim}


    \begin{Verbatim}[commandchars=\\\{\}]
this is a sentaence. it, now deleted by 0
[['this', 'is', 'a', 'sentaence', '.'], ['it', ',', 'now', 'deleted', 'by', '0']]

    \end{Verbatim}

    \begin{Verbatim}[commandchars=\\\{\}]
{\color{incolor}In [{\color{incolor}9}]:} \PY{c+c1}{\PYZsh{} from nltk.tokenize import sent\PYZus{}tokenize}
        
        \PY{c+c1}{\PYZsh{} string = \PYZdq{}this is a sentaence. it, now deleted by 0\PYZdq{}}
        
        \PY{c+c1}{\PYZsh{} sent\PYZus{}tokenize\PYZus{}list = sent\PYZus{}tokenize(string)}
        \PY{c+c1}{\PYZsh{} print(sent\PYZus{}tokenize\PYZus{}list)}
\end{Verbatim}


    \begin{Verbatim}[commandchars=\\\{\}]
['this is a sentaence.', 'it, now deleted by 0']

    \end{Verbatim}

    \subsection{Question 12 - 16*}\label{question-12---16}

    Project Euler is a website with mathematical problems that should/could
be solved by computers. Go to https://projecteuler.net/archives and
solve any 4 problems using Python.

As an example the problem number 16 can be solved in one line of Python:

sum(map(int, list(str(2**1000))))

1366

    \subsubsection{Problem 1}\label{problem-1}

If we list all the natural numbers below 10 that are multiples of 3 or
5, we get 3, 5, 6 and 9. The sum of these multiples is 23.

Find the sum of all the multiples of 3 or 5 below 1000.

    \begin{Verbatim}[commandchars=\\\{\}]
{\color{incolor}In [{\color{incolor}11}]:} \PY{k}{def} \PY{n+nf}{naturalNumbersMult}\PY{p}{(}\PY{p}{)}\PY{p}{:}                                       \PY{c+c1}{\PYZsh{} define function}
             \PY{n}{sum1} \PY{o}{=} \PY{l+m+mi}{0}                                                    \PY{c+c1}{\PYZsh{} set variable to 0}
             \PY{k}{for} \PY{n}{i} \PY{o+ow}{in} \PY{n+nb}{range}\PY{p}{(}\PY{l+m+mi}{1}\PY{p}{,}\PY{l+m+mi}{1000}\PY{p}{)}\PY{p}{:}                                     \PY{c+c1}{\PYZsh{} iteate from pre defined numbers}
                 \PY{k}{if} \PY{n}{i}\PY{o}{\PYZpc{}}\PY{k}{3} == 0 or i\PYZpc{}5 == 0:                                \PYZsh{} check if divisible by 3 or 5
                     \PY{n}{sum1} \PY{o}{+}\PY{o}{=} \PY{n}{i}                                           \PY{c+c1}{\PYZsh{} if it is then add it to sum}
             \PY{k}{return} \PY{n}{sum1}                                                 \PY{c+c1}{\PYZsh{} return the result }
\end{Verbatim}


    \begin{Verbatim}[commandchars=\\\{\}]
{\color{incolor}In [{\color{incolor}12}]:} \PY{n+nb}{print}\PY{p}{(}\PY{n}{naturalNumbersMult}\PY{p}{(}\PY{p}{)}\PY{p}{)}                                     \PY{c+c1}{\PYZsh{} display result}
\end{Verbatim}


    \begin{Verbatim}[commandchars=\\\{\}]
233168

    \end{Verbatim}

    \subsubsection{Problem 2}\label{problem-2}

    Each new term in the Fibonacci sequence is generated by adding the
previous two terms. By starting with 1 and 2, the first 10 terms will
be:

1, 2, 3, 5, 8, 13, 21, 34, 55, 89, ...

By considering the terms in the Fibonacci sequence whose values do not
exceed four million, find the sum of the even-valued terms.

    \begin{Verbatim}[commandchars=\\\{\}]
{\color{incolor}In [{\color{incolor}16}]:} \PY{k}{def} \PY{n+nf}{evenFib}\PY{p}{(}\PY{p}{)}\PY{p}{:}                                                  \PY{c+c1}{\PYZsh{} define function                   }
             \PY{n}{fib\PYZus{}list} \PY{o}{=} \PY{p}{[}\PY{l+m+mi}{1}\PY{p}{,}\PY{l+m+mi}{2}\PY{p}{]}                                            \PY{c+c1}{\PYZsh{} define array }
             \PY{n}{i} \PY{o}{=} \PY{l+m+mi}{1}
             \PY{n}{sum1} \PY{o}{=} \PY{l+m+mi}{2}
             \PY{k}{while} \PY{k+kc}{True}\PY{p}{:}                                                 \PY{c+c1}{\PYZsh{} run forever till stopped }
                 \PY{n}{fib\PYZus{}list}\PY{o}{.}\PY{n}{append}\PY{p}{(}\PY{n}{fib\PYZus{}list}\PY{p}{[}\PY{n}{i}\PY{o}{\PYZhy{}}\PY{l+m+mi}{1}\PY{p}{]} \PY{o}{+} \PY{n}{fib\PYZus{}list}\PY{p}{[}\PY{n}{i}\PY{p}{]}\PY{p}{)}
                 \PY{k}{if} \PY{n}{fib\PYZus{}list}\PY{p}{[}\PY{n}{i}\PY{o}{+}\PY{l+m+mi}{1}\PY{p}{]} \PY{o}{\PYZgt{}} \PY{l+m+mi}{4000000}\PY{p}{:}                             \PY{c+c1}{\PYZsh{} check condition }
                     \PY{k}{break}                                               \PY{c+c1}{\PYZsh{} break loop}
                 \PY{k}{if} \PY{n}{fib\PYZus{}list}\PY{p}{[}\PY{n}{i}\PY{o}{+}\PY{l+m+mi}{1}\PY{p}{]}\PY{o}{\PYZpc{}}\PY{k}{2} == 0:                                \PYZsh{} check if digit divisible by 0
                     \PY{n}{sum1} \PY{o}{+}\PY{o}{=} \PY{n}{fib\PYZus{}list}\PY{p}{[}\PY{n}{i}\PY{o}{+}\PY{l+m+mi}{1}\PY{p}{]}                               \PY{c+c1}{\PYZsh{} add to sum if divisible}
                 \PY{n}{i} \PY{o}{+}\PY{o}{=}\PY{l+m+mi}{1}                                                    
             \PY{k}{return} \PY{p}{(}\PY{n}{fib\PYZus{}list}\PY{p}{,} \PY{n}{sum1}\PY{p}{)}                                     \PY{c+c1}{\PYZsh{} return result}
\end{Verbatim}


    \begin{Verbatim}[commandchars=\\\{\}]
{\color{incolor}In [{\color{incolor}23}]:} \PY{n+nb}{print}\PY{p}{(}\PY{l+s+s2}{\PYZdq{}}\PY{l+s+s2}{the terms of the fib list are : }\PY{l+s+se}{\PYZbs{}n}\PY{l+s+s2}{ }\PY{l+s+si}{\PYZob{}\PYZcb{}}\PY{l+s+s2}{ }\PY{l+s+s2}{\PYZdq{}}\PY{o}{.}\PY{n}{format}\PY{p}{(}\PY{n}{evenFib}\PY{p}{(}\PY{p}{)}\PY{p}{[}\PY{l+m+mi}{0}\PY{p}{]}\PY{p}{)}\PY{p}{)}     \PY{c+c1}{\PYZsh{} display all the terms }
         \PY{n+nb}{print}\PY{p}{(}\PY{l+s+s2}{\PYZdq{}}\PY{l+s+se}{\PYZbs{}n}\PY{l+s+s2}{the sum of even valued terms : }\PY{l+s+si}{\PYZob{}\PYZcb{}}\PY{l+s+s2}{\PYZdq{}}\PY{o}{.}\PY{n}{format}\PY{p}{(}\PY{n}{evenFib}\PY{p}{(}\PY{p}{)}\PY{p}{[}\PY{l+m+mi}{1}\PY{p}{]}\PY{p}{)}\PY{p}{)}        \PY{c+c1}{\PYZsh{} display the total result }
\end{Verbatim}


    \begin{Verbatim}[commandchars=\\\{\}]
the terms of the fib list are : 
 [1, 2, 3, 5, 8, 13, 21, 34, 55, 89, 144, 233, 377, 610, 987, 1597, 2584, 4181, 6765, 10946, 17711, 28657, 46368, 75025, 121393, 196418, 317811, 514229, 832040, 1346269, 2178309, 3524578, 5702887] 

the sum of even valued terms : 4613732

    \end{Verbatim}

    \subsubsection{Problem 3}\label{problem-3}

    The prime factors of 13195 are 5, 7, 13 and 29.

What is the largest prime factor of the number 600851475143 ?

    \begin{Verbatim}[commandchars=\\\{\}]
{\color{incolor}In [{\color{incolor}2}]:} \PY{k+kn}{import} \PY{n+nn}{math}                                                       \PY{c+c1}{\PYZsh{} import library}
\end{Verbatim}


    \begin{Verbatim}[commandchars=\\\{\}]
{\color{incolor}In [{\color{incolor}3}]:} \PY{k}{def} \PY{n+nf}{isPrimeNumber}\PY{p}{(}\PY{n}{n}\PY{p}{)}\PY{p}{:}                                             \PY{c+c1}{\PYZsh{} define function and check conditoins}
            \PY{k}{if} \PY{n}{n} \PY{o}{\PYZlt{}}\PY{o}{=} \PY{l+m+mi}{1}\PY{p}{:}
                \PY{k}{return} \PY{k+kc}{False}
            \PY{k}{elif} \PY{n}{n} \PY{o}{\PYZlt{}}\PY{o}{=} \PY{l+m+mi}{3}\PY{p}{:}
                \PY{k}{return} \PY{k+kc}{True}
            \PY{k}{elif} \PY{n}{n} \PY{o}{\PYZpc{}} \PY{l+m+mi}{2} \PY{o}{==} \PY{l+m+mi}{0} \PY{o+ow}{or} \PY{n}{n} \PY{o}{\PYZpc{}} \PY{l+m+mi}{3} \PY{o}{==} \PY{l+m+mi}{0}\PY{p}{:}
                \PY{k}{return} \PY{k+kc}{False}
            \PY{n}{i} \PY{o}{=} \PY{l+m+mi}{5}
            \PY{k}{while} \PY{n}{i}\PY{o}{*}\PY{n}{i} \PY{o}{\PYZlt{}}\PY{o}{=} \PY{n}{n}\PY{p}{:}
                \PY{k}{if} \PY{n}{n}\PY{o}{\PYZpc{}}\PY{k}{i}==0 or n\PYZpc{}(i+2)==0:                     
                    \PY{k}{return} \PY{k+kc}{False}
                \PY{n}{i} \PY{o}{+}\PY{o}{=}\PY{l+m+mi}{6}
            \PY{k}{return} \PY{k+kc}{True}                                                   \PY{c+c1}{\PYZsh{} return respective conditon}
\end{Verbatim}


    \begin{Verbatim}[commandchars=\\\{\}]
{\color{incolor}In [{\color{incolor}4}]:} \PY{k}{def} \PY{n+nf}{largestPrimeFactor}\PY{p}{(}\PY{n}{n}\PY{p}{)}\PY{p}{:}                                        \PY{c+c1}{\PYZsh{} define function}
            \PY{n}{largest\PYZus{}prime} \PY{o}{=} \PY{l+m+mi}{0}                                             \PY{c+c1}{\PYZsh{} set varibale to 0}
            \PY{k}{for} \PY{n}{i} \PY{o+ow}{in} \PY{n+nb}{range}\PY{p}{(}\PY{l+m+mi}{1}\PY{p}{,}\PY{n+nb}{int}\PY{p}{(}\PY{n}{math}\PY{o}{.}\PY{n}{sqrt}\PY{p}{(}\PY{n}{n}\PY{p}{)}\PY{p}{)}\PY{o}{\PYZhy{}}\PY{l+m+mi}{1}\PY{p}{)}\PY{p}{:}                        \PY{c+c1}{\PYZsh{} iterate the provided number(sqrt)}
                \PY{k}{if} \PY{n}{isPrimeNumber}\PY{p}{(}\PY{n}{i}\PY{p}{)} \PY{o+ow}{and} \PY{n}{n}\PY{o}{\PYZpc{}}\PY{k}{i}==0:                           \PYZsh{} check if the number is prime using defined function
                    \PY{n}{largest\PYZus{}prime} \PY{o}{=} \PY{n}{i}
            \PY{k}{return} \PY{n}{largest\PYZus{}prime}                                          \PY{c+c1}{\PYZsh{} return result}
\end{Verbatim}


    \begin{Verbatim}[commandchars=\\\{\}]
{\color{incolor}In [{\color{incolor}6}]:} \PY{n+nb}{print}\PY{p}{(}\PY{n}{largestPrimeFactor}\PY{p}{(}\PY{l+m+mi}{600851475143}\PY{p}{)}\PY{p}{)}                           \PY{c+c1}{\PYZsh{} display result}
\end{Verbatim}


    \begin{Verbatim}[commandchars=\\\{\}]
6857

    \end{Verbatim}

    \subsubsection{Problem 4}\label{problem-4}

    A palindromic number reads the same both ways. The largest palindrome
made from the product of two 2-digit numbers is 9009 = 91 × 99.

Find the largest palindrome made from the product of two 3-digit
numbers.

    \begin{Verbatim}[commandchars=\\\{\}]
{\color{incolor}In [{\color{incolor}50}]:} \PY{k}{def} \PY{n+nf}{isPalindrome}\PY{p}{(}\PY{n}{n}\PY{p}{)}\PY{p}{:}                                               \PY{c+c1}{\PYZsh{} define function   }
             \PY{k}{if} \PY{n+nb}{str}\PY{p}{(}\PY{n}{n}\PY{p}{)}\PY{o}{==}\PY{n+nb}{str}\PY{p}{(}\PY{n}{n}\PY{p}{)}\PY{p}{[}\PY{p}{:}\PY{p}{:}\PY{o}{\PYZhy{}}\PY{l+m+mi}{1}\PY{p}{]}\PY{p}{:}                                       \PY{c+c1}{\PYZsh{} check if string is a palandrome}
                 \PY{k}{return} \PY{k+kc}{True}                                                \PY{c+c1}{\PYZsh{} return true is palandrome  }
             \PY{k}{else}\PY{p}{:}
                 \PY{k}{return} \PY{k+kc}{False}
\end{Verbatim}


    \begin{Verbatim}[commandchars=\\\{\}]
{\color{incolor}In [{\color{incolor}55}]:} \PY{k}{def} \PY{n+nf}{largestProductPalindrome}\PY{p}{(}\PY{p}{)}\PY{p}{:}                                    \PY{c+c1}{\PYZsh{} define function }
             \PY{n}{max\PYZus{}product} \PY{o}{=} \PY{l+m+mi}{0}                                                \PY{c+c1}{\PYZsh{} set variable to 0 }
             \PY{k}{for} \PY{n}{i} \PY{o+ow}{in} \PY{n+nb}{range}\PY{p}{(}\PY{l+m+mi}{100}\PY{p}{,}\PY{l+m+mi}{1000}\PY{p}{)}\PY{p}{:}                                      \PY{c+c1}{\PYZsh{} iterate through provided range}
                 \PY{k}{for} \PY{n}{j} \PY{o+ow}{in} \PY{n+nb}{range}\PY{p}{(}\PY{l+m+mi}{100}\PY{p}{,}\PY{l+m+mi}{1000}\PY{p}{)}\PY{p}{:}
                     \PY{n}{product} \PY{o}{=} \PY{n}{i} \PY{o}{*} \PY{n}{j}
                     \PY{k}{if} \PY{n}{product} \PY{o}{\PYZgt{}} \PY{n}{max\PYZus{}product} \PY{o+ow}{and} \PY{n}{isPalindrome}\PY{p}{(}\PY{n}{product}\PY{p}{)}\PY{p}{:}    \PY{c+c1}{\PYZsh{} check if palandrome using defined function  }
                         \PY{n}{max\PYZus{}product} \PY{o}{=} \PY{n}{product}                              \PY{c+c1}{\PYZsh{} update if value is greater }
             \PY{k}{return} \PY{n}{max\PYZus{}product}                                             \PY{c+c1}{\PYZsh{} return result }
\end{Verbatim}


    \begin{Verbatim}[commandchars=\\\{\}]
{\color{incolor}In [{\color{incolor}56}]:} \PY{n+nb}{print}\PY{p}{(}\PY{n}{largestProductPalindrome}\PY{p}{(}\PY{p}{)}\PY{p}{)}                                  \PY{c+c1}{\PYZsh{} display result}
\end{Verbatim}


    \begin{Verbatim}[commandchars=\\\{\}]
906609

    \end{Verbatim}


    % Add a bibliography block to the postdoc
    
    
    
    \end{document}
